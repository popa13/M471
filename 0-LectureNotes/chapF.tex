
\chapter{Moments}

In this chapter, we assume that a probability space $(S, \mathcal{A}, P)$ is given.

\section{Moments}

\begin{definition}
If $X$ is a random variable, then the $k$-th moment, with $k \geq 0$ an integer, is given by
    \[
        M_k := \mathrm{Exp} (X^k ) ,
    \]
if the expectation exists.
\end{definition}

When $X$ is a continuous random variable with pdf $f_X$, then
    \[
        M_k = \int_{-\infty}^\infty x^k f_X (x) \, dx .
    \]

\begin{example}
If $X \sim U (a, b)$, then
    \[
        M_k = \int_a^b x^k \, dx = \frac{b^{k + 1} - a^{k + 1}}{k + 1} . \tag*{$\triangle$}
    \]
\end{example}

\begin{definition}
If $X$ is a random variable with mean $\mu_X$, then the $k$-th centered moment, with $k \geq 0$ an integer is given by
    \[
        CM_k := \mathrm{Exp} ( (X - \mu_X)^k) ,
    \]
if the expected value exists.
\end{definition}

The variance of $X$ is the second centered moment of a random variable. Notice also that the $\mathrm{Var} (X)$ can be rewritten as a polynomial expression involving only $M_k$:
    \[
        \mathrm{Var} (X) = \mathrm{Exp} (X^2) - (\mathrm{Exp} (X))^2 = M_2 - (M_1)^2 .
    \]

\section{Moment Generating Function}

Notice that, if all the moments of a continuous random variable $X$ exist, then
    \[
        \sum_{k = 0}^\infty \frac{t^k}{k!} \mathrm{Exp} (X^k) ,
    \]
may exists for a certain value of $t > 0$ (and therefore for any value less than $t$). If this happens, then
    \[
        \sum_{k = 0}^\infty \frac{t^k}{k!} \mathrm{Exp} (X^k) = \int_{-\infty}^\infty \sum_{k = 0}^\infty \frac{t^k x^k}{k!} f_X (x) \, dx = \int_{-\infty}^\infty e^{tx} f_X (x) \, dx = \mathrm{Exp} (e^{tX}) .
    \]
The expectation on the right-hand side is called the \textbf{moment generating function} of $X$ and is denoted by $M_X (t)$. 

\begin{example}
If $X$ has the exponential distribution with parameter $\lambda$, then $f_X (x) = \lambda e^{-\lambda x}$ for $x \geq 0$ so that
    \[
        M_X (t) = \int_0^\infty e^{tx} \lambda e^{-\lambda x} \, dx .
    \]
If $t < \lambda$, then $(t - \lambda ) x < 0$ for any $x > 0$. Therefore, the above integral does exists with $t < \lambda$ and
    \[
        M_X (t) = \frac{\lambda}{\lambda - t} .
    \]
When $t > \lambda$, then
    \[
        M_X (t) = \infty .
    \] 
The moment generating function $M_X$ only exists when $t < \lambda$. \hfill $\triangle$
\end{example}

\begin{example}
Ìf $X$ has the normal distribution with mean $0$ and variance $1$, then
    \[
        M_X (t) = \int_{-\infty}^\infty e^{tx} \frac{1}{\sqrt{2\pi}} e^{-\frac{1}{2} x^2} \, dx = e^{\frac{1}{2} t^2} .
    \]
Using the fact that $e^t = 1 + t + t^2/2 + t^3/3! + \ldots$, we obtain
    \[
        M_X (t) = 1 + \frac{1}{2} t^2 + \frac{1}{8} t^4 + \frac{1}{8 \cdot 3!} t^6 + \ldots . \tag*{$\triangle$} 
    \]
\end{example}

The moment generating function is really useful to find the moments of a random variable.

\begin{theorem}\label{T:MomentsAndMomentGeneratingFunction}
If $M_X (t)$ exists in a neighborhood of $0$, then, for $k = 1 , 2 , \ldots$
    \[
        \mathrm{Exp} (X^k) = M_X^{(k)} (0).
    \]
\end{theorem}

\begin{example}
Let $X$ be a random variable with the exponentital distribution of parameter $\lambda >0$. Compute $M_3$. 
\end{example}

\begin{sol*}
From the previous examples, we have $M_X (t) = \frac{\lambda}{\lambda - t}$, for $t < \lambda$. We have
    \[
        M_X'(t) = \frac{\lambda}{(\lambda-t)^2}, \quad M_X'' (t) = \frac{2 \lambda}{(\lambda - t)^3} \quad \text{ and } \quad M_X^{(3)} (t) = \frac{6\lambda}{(\lambda - t)^4} .
    \]
Therefore, 
    \[
        M_3 = M_X^{(3)} (0) = \frac{6\lambda}{(\lambda - 0)^4} = 6 \lambda^{-3} . \tag*{$\triangle$}
    \]
\end{sol*}

\begin{theorem}\label{Thm:MomentGeneratingOfSum}
If $X$ and $Y$ are independent random variables, then $X + Y$ has moment generating function
    \[
        M_{X + Y} (t) = M_X (t) M_Y (t) .
    \]
\end{theorem}
\begin{proof}
Since $X$ and $Y$ are independent, we have $f_{X, Y} (x, y) = f_X (x) f_Y (y)$. Hence, 
    \begin{align*}
        M_{X + Y} (t) = \mathrm{Exp} (e^{t (X + Y)}) &= \int_{-\infty}^\infty \int_{-\infty}^\infty e^{t (x + y)} f_{X, Y} (x, y) \, dx dy \\ 
        &= \int_{-\infty}^\infty \int_{-\infty}^\infty e^{tx} f_X (x) e^{ty} f_Y (y) \, dx dy \\ 
        &= \mathrm{Exp} (e^{tX}) \mathrm{Exp} (e^{tY}) . \qedhere
    \end{align*}
\end{proof}

\begin{theorem}
Let $X$ be a continuous random variable. Assume the generating function $M_X$ satisfies $M_X (t) = \mathrm{Exp} (e^{tX})$ for all $t$ with $-\delta < t \leq \delta$ and for some $\delta > 0$. Then
    \begin{enumerate}[label=\alph*)]
         \item There is a unique distribution with moment generating function $M_X$.
         \item Furthermore, under these conditions, we have that $\mathrm{Exp} (X^k) < \infty$ for any $k = 1, 2, \ldots$ and
            \[
                M_X (t) = \sum_{k = 0}^\infty \frac{t^k}{k!} \mathrm{Exp} (X^k ) \quad |t| < \delta .
            \]
     \end{enumerate} 
\end{theorem}

\begin{example}
Let $X$ and $Y$ be independent random variables, $X \sim N (\mu_X , \sigma_X )$ and $Y \sim N (\mu_Y , \sigma_Y )$. Show that their sum $Z = X + Y$ has the normal distribution with parameters $\mu_X + \mu_Y$ and $\sigma_X^2 + \sigma_Y^2$.
\end{example}

\begin{sol*}
Let $z = (x - \mu_X )/ \sigma_X$, then
    \begin{align*}
        M_X (t) &= \int_{-\infty}^\infty e^{tx} \frac{1}{\sqrt{2\pi \sigma_X^2}} e^{-\frac{1}{2\sigma_X^2} (x - \mu_X)^2} \, dx \\ 
        &= \int_{-\infty}^\infty e^{t (\sigma_X z + \mu_X)} \frac{1}{\sqrt{2\pi}} e^{-\frac{1}{2} z^2} \, dz \\ 
        &= e^{t \mu_X} \int_{-\infty}^\infty \frac{1}{\sqrt{2\pi}} e^{t \sigma_X z} e^{-\frac{1}{2} z^2} \, dz \\ 
        &= e^{t \mu_X + \frac{1}{2} t^2 \sigma_X^2} \int_{-\infty}^\infty \frac{1}{\sqrt{2\pi}} e^{-\frac{1}{2} (z - t \sigma_X)^2} \, dz
    \end{align*}
and hence $M_X (t) = \exp (t \mu_X + \frac{1}{2} t^2 \sigma_X^2 )$. Similarly, we have $M_Y (t) = \exp (t \mu_Y + \frac{1}{2} t^2 \sigma_Y^2 )$. Therefore, by Theorem \ref{Thm:MomentGeneratingOfSum},
    \[
        M_{X + Y} (t) = M_X (t) M_Y (t) = \exp \Big(t (\mu_X + \mu_Y) + \frac{1}{2} t^2 (\sigma_X^2 + \sigma_Y^2 )\Big) .
    \]
By uniqueness of the moment generating function, we see that 
    \begin{equation}
        X + Y \sim N (\mu_x + \mu_Y , \sigma_X^2 + \sigma_Y^2 ).  \tag*{$\triangle$}
    \end{equation}
\end{sol*}

\begin{comment}

\section{Problems Set}

\subsection*{Moments}

\begin{problem}
If $X$ is uniformly distributed on $(a, b)$, show that
    \[
        \mathrm{Exp} (X^k) = \frac{b^{k + 1} - a^{k + 1}}{(b - a) (k + 1)} \quad \text{ for } k = 1, 2, \ldots .
    \]
\end{problem}

\subsection*{Moment Generating Function}

\begin{problem}
If $X$ has the normal distribution with mean $0$ and variance $1$, find $\mathrm{E} (X^3)$. 
\end{problem}

\begin{problem}
Show that, if $X$ has a normal distribution, then so does $aX + b$, for any $a, b \in \mR$ with $a \neq 0$.
\end{problem}

%\begin{problem}
%Identify the distribution of the random variables with the following moment-generating functions:
%   \begin{enumerate}[label=\alph*)]
%   \item $M (t) = (1 - 4t)^{-2}$.
%   \item $M (t) = 1/ (1 - 3.2t)$.
%   \item $M (t) = e^{-5t + 6t^2}$. 
%   \end{enumerate}
%\end{problem}

\begin{problem}
Suppose that the waiting time for the first customer to enter a retail shop after 9:00\textsc{am} is a random variable $X$ with an exponential density function given by
    \[
        f(x) = \left\{ \begin{matrix} (1/\theta ) e^{-x/\theta} & x > 0 , \\ 
        0 & \text{ elsewhere.} \end{matrix} \right.
    \]
    \begin{enumerate}[label=\alph*)]
    \item Find the moment-generating function of $X$.
    \item Use the answer from part (a) to find $\mathrm{Exp} (X)$ and $\mathrm{Var} (X)$.
    \end{enumerate}
\end{problem}



\section{Solutions to Problems Set}
\setcounter{problem}{0} 

\subsection*{Moments}

\begin{problem}
From the definition of the $k$-th moment, we compute
    \[
        \mathrm{Exp} (X^k) = \int_{-\infty}^\infty x^k f_X (x) \, dx = \int_a^b \frac{x^k}{b - a} \, dx .
    \]
After integrating $x^k$, we obtain the answer:
    \[
        \mathrm{Exp} (X^k) = \frac{b^{k + 1} - a^{k + 1}}{(k + 1) (b - a)} . \tag*{$\triangle$}
    \]
\end{problem}

\subsection*{Moment Generating Function}

\begin{problem}
The moment generating function of the normal distribution is
    \[
        M_X (t) = e^{\mu_X t + \frac{1}{2}\sigma_X^2 t^2} .
    \]
Since $\mu_X = 0$ and $\sigma_X^2 = 1$, we then get $M_X (t) = e^{t^2/2}$. From Theorem \ref{T:MomentsAndMomentGeneratingFunction}, we know that
    \[
        \mathrm{Exp} (X^3 ) = M_X^{(3)} (0) .
    \]
We have
    \[
        M_X^{(3)} (t) = e^{t^2/2} t (t^2 + 3) \quad \Rightarrow \quad \mathrm{Exp} (X^3) = 0 .\tag*{$\triangle$}
    \]
\end{problem}

\begin{problem}
We will identify the moment generating function of $aX + b$. From the definition of the moment generating function, we have
    \[  
        M_{aX + b} (t) = \mathrm{Exp} (e^{t(aX + b)}) = e^{tb} \mathrm{Exp} (e^{taX}) = e^{tb} M_X (at ) .
    \]
Since $X \sim N (\mu_X , \sigma_X )$, we know that $M_X (t) = e^{\mu_X t + \frac{1}{2} \sigma_X^2 t^2}$. Substituting this into the equation for $M_{aX + b} (t)$, we find that
    \[
        M_{aX + b} (t) = e^{tb} e^{\mu_X at + \frac{1}{2} \sigma_X^2 a^2 t^2} = e^{(b + \mu_X a) t + \frac{1}{2} \sigma_X^2 a^2 t^2} .
    \]
The moment generating function of $aX + b$ is then the moment generating function of a normal distribution with average $b + \mu_X a$ and variance $\sigma_X^2 a^2$. Hence, $aX + b \sim N (b + \mu_X a , \sigma_X^2 a^2 )$. \hfill $\triangle$
\end{problem}

%\begin{problem}
%Identify the distribution of the random variables with the following moment-generating functions:
%   \begin{enumerate}[label=\alph*)]
%   \item $M (t) = (1 - 4t)^{-2}$.
%   \item $M (t) = 1/ (1 - 3.2t)$.
%   \item $M (t) = e^{-5t + 6t^2}$. 
%   \end{enumerate}
%\end{problem}

\begin{problem}
    \begin{enumerate}[label=\alph*)]
        \item Since $X$ is an exponential distribution and $\lambda = 1/\theta$, we find that
            \[
                M_X (t) = \frac{1/\theta}{1/\theta - t} = \frac{1}{1 - \theta t} , 
            \]
        for $t < 1/\theta$.
        \item We know that
            \[
                \mathrm{Exp} (X) = M'_X (0) \quad \text{ and } \quad \mathrm{Var} (X) = \mathrm{Exp} (X^2) - (\mathrm{Exp} (X))^2 = M''_X (0) - (M'_X(0))^2 .
            \]
        We compute
            \[
                M_X' (t) = \frac{\theta}{(1 - \theta t)^2} \quad \text{ and } \quad M_X'' (t) = \frac{2\theta^2}{(1 - \theta t)^{3}}
            \]
        Hence
            \[
                \mathrm{Exp} (X) = \frac{\theta}{(1 - \theta (0))^2} = \theta
            \]
        and
            \[
                \mathrm{Var} (X) = \frac{2\theta^2}{(1 - \theta (0))^3} - (\theta)^2 = \theta^2 . \tag*{$\triangle$}
            \]
    \end{enumerate}
\end{problem}

\end{comment}