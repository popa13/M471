\documentclass[12pt]{amsart}

\usepackage{xcolor}
\usepackage{hyperref}
\usepackage[margin=2.2cm, headsep=1cm, footskip=1cm]{geometry}
\usepackage{footmisc}
\usepackage{multicol}

\usepackage{fancyhdr}
\pagestyle{fancy}
\fancyhf{}
\renewcommand{\headrulewidth}{2pt}
\renewcommand{\footrulewidth}{0pt}
\rfoot{\thepage}
\lhead{\textsc{Math} 471}
\chead{\textsc{Syllabus}}
\rhead{Fall 2023}

\newcommand{\spacer}{\vspace{.2cm}}
\newcommand{\svs}{\vspace{.1cm}}

\newcommand{\red}[1]{\textcolor{red}{#1}}
\definecolor{gold}{rgb}{0.80,0.68,0.00}\newcommand{\gold}[1]{\textcolor{gold}{#1}}

\begin{document}
\thispagestyle{empty}

\begin{center}
\textsc{Math 471} \hfill {\Large\textsc{Syllabus}} \hfill \textsc{Fall 2023}
\end{center}

\noindent\hrulefill

\noindent \textbf{Lecture}: MWF 1:30--2:20pm \\
Classroom: Keller 403

\spacer

\noindent\textbf{Instructor:} Pierre-Olivier Paris{\'e} (email: \texttt{parisepo@hawaii.edu})\\
Office: Physical Science Building (PSB) 302\\
Office hours: TBA

\noindent\hrulefill

\section*{Course description}
Probability spaces, random variables, distributions, expectations, moment-generating and characteristic functions, limit theorems. Continuous probability emphasized.\svs

\section*{Course objectives}
Upon successful completion, the student will have a foundation in the basic topics of the theory of Probability listed above in the syllabus. Emphasis on rigor will provide students the understanding needed for graduate work, and in the study of the logical foundations of mathematics.

\noindent{\bf Prerequisites:}
Math 244 (or concurrent) or 253A (or concurrent); recommended 305 or 371 or 372.

\section*{Required course material}
\noindent{\bf Course website:} \url{https://mathopo.ca/courses-website/MATH-471/MATH-471.html}. All the information about the course (like the schedule, important dates) is posted on the course website. Assignments will be posted on the Laulima website.

\section*{Important Dates}
%Please make sure you write down somewhere these important dates:
Refer to the schedule on the course website.
	\begin{itemize}
	\item Final:
		\begin{itemize}
		\item December, 15th, 2:15--4:15pm.
		\end{itemize}
	\item Non-instructional day(s):
		\begin{itemize}
		\item Labor Day, September 4th.
		\item Veteran's Day, November 10th.
		\item The day after Thanksgiving Day, November 24th.
		\end{itemize}
	\end{itemize}

\section*{Grading components}
Your final grade will be calculated based on a weighted average of the following components.
\begin{enumerate}
\item{\bf Homework (60\%):} There will be homework assigned at the beginning of each chapter and due a week after the chapter is finished\footnote{There are two exceptions. The homework for Appendix L and Appendix S will be due on Monday, August 28th.}. The problems will be posted on Laulima under the Assignments tab. Your homework should be submitted in my mailbox in Keller 418.
\item {\bf Final exam (40\%):} There will be a final comprehensive exam as scheduled by the university on December, 15th, 2:15--4:15pm.
\end{enumerate}

\section*{Lectures}
If you miss a lecture, you are responsible for any assignments, announcements made, and lecture notes missed. Unavoidable absences should be explained to the instructor. Office hours will not be utilized to present a previous lecture missed.

\section*{Missed assignment policies}

\noindent{\bf Policies for exams:} The final exam cannot be taken before the scheduled time for any reason.

\noindent{\bf Policies for homework:} Late homework won't be accepted and will result of a mark of zero(0).
\svs

\noindent{\bf Academic integrity:}
All students are expected to abide by the university's Conduct Code. Academic sanctions for dishonesty may include receiving an F in the assignment or receiving an F in the class. There may be additional administrative sanctions.

\section*{Classroom policies}
Tablets can be used for note-taking only. During lectures, to respect the lecturer and your classmates, refrain from using electronic items, cell phones, music players, tablets, and laptops for any other purposes than note-taking.

Please arrive, be seated and ready to start each class on time. If you have a valid reason to leave early, please advise me before the class and try to sit near the exit to minimize disruption.

\section*{Sources of help}
{\bf KOKUA:} I am happy to work with you and the KOKUA Program (Office for Students with Disabilities), if you need course accommodations due to a disability. KOKUA can be reached at (808) 956-7511 or (808) 956-7612 (voice/text) in room 013 of the Queen Lili`uokalani Center for Student Services. All course modifications must be arranged through KOKUA. You are encouraged to start this process as early as possible.

\section*{Concerns}
If at any time during the semester you have any questions or concerns about the class, please contact me during regularly scheduled office hours or via email to make an appointment. You may also contact the following people:
\spacer

\noindent {\bf Director of Undergraduate Studies}\\
Mirjana Jovovic \\
Email: \texttt{undergrad-dir@math.hawaii.edu}

\svs
\noindent {\bf Associate Chair}\\
Bj{\o}rn Kjos-Hanssen \\
Email: \texttt{assoc-chair@math.hawaii.edu}

\end{document}

