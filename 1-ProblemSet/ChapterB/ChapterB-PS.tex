\documentclass[12pt]{article}
\usepackage[utf8]{inputenc}

\usepackage{lmodern}

\usepackage{enumitem}
\usepackage[margin=2cm]{geometry}

\usepackage{amsmath, amsfonts, amssymb}
\usepackage{graphicx}
\usepackage{subfigure}
\usepackage{tikz}
\usepackage{pgfplots}
\usepackage{multicol}

\usepackage{titlesec}
\usepackage{environ}
\usepackage{xcolor}
\usepackage{fancyhdr}
\usepackage[colorlinks = true, linkcolor = black]{hyperref}
\usepackage{xparse}
\usepackage{enumitem}
\usepackage{comment}
\usepackage{wrapfig}
\usepackage[capitalise]{cleveref}
\usepackage{epsdice}
\usepackage{circledsteps}

\usepackage{url}
\usepackage{calc}
%\usepackage{subcaption}
\usepackage[indent=0pt]{parskip}

\usepackage{array}
\usepackage{blkarray,booktabs, bigstrut}
\usepackage{bigints}

\pgfplotsset{compat=1.16}

% MATH commands
\newcommand{\ga}{\left\langle}
\newcommand{\da}{\right\rangle}
\newcommand{\oa}{\left\lbrace}
\newcommand{\fa}{\right\rbrace}
\newcommand{\oc}{\left[}
\newcommand{\fc}{\right]}
\newcommand{\op}{\left(}
\newcommand{\fp}{\right)}

\newcommand{\bi}{\mathbf{i}}
\newcommand{\bj}{\mathbf{j}}
\newcommand{\bk}{\mathbf{k}}
\newcommand{\bF}{\mathbf{F}}

\newcommand{\mR}{\mathbb{R}}
\newcommand{\mC}{\mathbb{C}}
\newcommand{\mT}{\mathbb{T}}
\newcommand{\mD}{\mathbb{D}}

\newcommand{\ra}{\rightarrow}
\newcommand{\Ra}{\Rightarrow}

\newcommand{\sech}{\mathrm{sech}\,}
\newcommand{\csch}{\mathrm{csch}\,}
\newcommand{\curl}{\mathrm{curl}\,}
\newcommand{\dive}{\mathrm{div}\,}

\newcommand{\ve}{\varepsilon}
\newcommand{\spc}{\vspace*{0.5cm}}

\DeclareMathOperator{\Ran}{Ran}
\DeclareMathOperator{\Dom}{Dom}
\DeclareMathOperator{\re}{Re}
\DeclareMathOperator{\im}{Im}
%\DeclareMathOperator{\arg}{arg}

\usepackage{pifont}

\newcommand{\club}{\ding{168}}
\newcommand{\spade}{\ding{171}}
\newcommand{\ddiamond}{{\color{red}\ding{169}}}
\newcommand{\heart}{{\color{red}\ding{170}}}

% Playing card
\newcommand{\cardH}[2]{%
\begin{tikzpicture}[trim right = #2pt, scale=0.1]
  % Card outline
  \draw[thick, red] (0,0) rectangle (5,3.5);
  % Card suit and value
  \node at (1.4,1.75) {\tiny\color{red} #1};
  \node[red] at (3.5, 1.75) {\scriptsize\heart};
\end{tikzpicture}
}
\newcommand{\cardD}[2]{%
\begin{tikzpicture}[trim right = #2pt, scale=0.1]
  % Card outline
  \draw[thick, red] (0,0) rectangle (5,3.5);
  % Card suit and value
  \node at (1,1.75) {\tiny\color{red} #1};
  \node[red] at (3.5, 1.75) {\footnotesize\ddiamond};
\end{tikzpicture}
}
\newcommand{\cardS}[2]{%
\begin{tikzpicture}[trim right = #2pt, scale=0.1]
  % Card outline
  \draw[thick, black] (0,0) rectangle (5,3.5);
  % Card suit and value
  \node at (1,1.75) {\tiny\color{black} #1};
  \node[black] at (3.5, 1.75) {\footnotesize\spade};
\end{tikzpicture}
}
\newcommand{\cardC}[2]{%
\begin{tikzpicture}[trim right = #2pt, scale=0.1]
  % Card outline
  \draw[thick, black] (0,0) rectangle (5,3.5);
  % Card suit and value
  \node at (1,1.75) {\tiny\color{black} #1};
  \node[black] at (3.5, 1.75) {\footnotesize\club};
\end{tikzpicture}
}

%% Defining example environment
\newcounter{totNumProblems}
\newcounter{problem}[section]
\NewEnviron{problem}%
	{%
	\noindent\refstepcounter{problem}\refstepcounter{totNumProblems}\fcolorbox{gray!40}{gray!40}{\textsc{\textcolor{black}{Problem~\theproblem.}}}%
	%\fcolorbox{black}{white}%
		{  %\parbox{0.95\textwidth}%
			{
			\BODY
			}%
		}%
	}


%% Redefining sections
\newcommand{\sectionformat}[1]{%
    \begin{tikzpicture}[baseline=(title.base)]
        \node[rectangle, draw] (title) {#1 \thesection};
    \end{tikzpicture}
    
    \noindent\hrulefill
}

\renewcommand{\thesection}{\Alph{section}}
\renewcommand{\thesubsection}{\Alph{section}.\arabic{subsection}}

% default values copied from titlesec documentation page 23
% parameters of \titleformat command are explained on page 4
\titleformat{\section}{\centering\normalfont\large\scshape}{}{1em}{\centering\sectionformat}

%% Set counters for sections to none
%\setcounter{secnumdepth}{0}

%% Set the footer/headers
\pagestyle{fancy}
\fancyhf{}
\renewcommand{\headrulewidth}{0pt}
\renewcommand{\footrulewidth}{2pt}
\lfoot{P.-O. Paris{\'e}}
\cfoot{MATH 471}
\rfoot{Page \thepage}

\begin{document}
\hrulefill

\begin{minipage}{0.33\textwidth}
\textsc{Math 471}
\end{minipage} \hfill 
\begin{minipage}{0.32\textwidth}
\centering
\textsc{Problems Set} \\
Chapter B
\end{minipage}
 \hfill 
 \begin{minipage}{0.33\textwidth}
 \flushright \textsc{Fall 2023}
 \end{minipage}

\hrulefill

\setcounter{section}{2}

\subsection{Conditional Probabilities}
	
	\begin{problem}
	Let $(S, \mathcal{A} , P )$ be a probability space. Suppose two events $A$ and $B$ are given such that $P (A) = 0.5$, $P(B) = 0.3$, and $P (A \cap B) = 0.1$. Find
		\begin{enumerate}[label=\alph*)]
		\begin{multicols}{3}
		\item $P(A|B)$.
		\item $P (A | A \cup B)$.
		\item $P (A \cap B | A \cup B )$.
		\end{multicols}
		\end{enumerate}			
	\end{problem}

	\begin{problem}
	A balanced die is tossed once. What is the probability the die lands on a $1$, given that an odd number was obtained?
	\end{problem}

	\begin{problem}
	Two fair dice are rolled. What is the probability that at least one lands on 6 given that the dice land on different numbers?
	\end{problem}

	\begin{problem}
	Let $(S , \mathcal{A} , P)$ be a probability space. Suppose that two events $A$ and $B$ are given such that $P (A) > 0$, $P (B) > 0$. Prove that if $P (A) < P (A | B)$, then $P (B) < P (B | A)$.
	\end{problem}
	
	\begin{problem}
	Suppose that $A \subset B$ and that $P (A) > 0$ and $P (B) > 0$. Show that $P (B | A) = 1$ and $P (A | B) = P (A) / P (B)$.
	\end{problem}
	
	\begin{problem}
	If $A$ and $B$ are mutually exclusive events and $P (B) > 0$, show that
	\begin{align*}
	P (A | A \cup B) = \frac{P (A)}{P (A) + P (B)} .
	\end{align*}
	\end{problem}
	
	\begin{problem}
	Let $(S, \mathcal{A}, P )$ be a probability space. If $A, B$ are events with $P (A) > 0$ and $P (B) > 0$, then show that
		\begin{align*}
		\frac{P (A|B)}{P (\overline{A} | B)} = \frac{P (A)}{P (\overline{A})} \frac{P (B|A)}{P (B|\overline{A})} .
		\end{align*}
	\end{problem}
	
	\subsection{Bayes' Formula}
	
	\begin{problem}
	A laboratory blood test is 95\% effective in detecting a certain disease when it is, in fact, present. However, the test also yields a ``false positive'' result for 1\% of the healthy people tested\footnote{That is, if a healthy person is tested, then, with probability 0.01, the test result will imply the person has the disease.}. If 0.5\% of the population actually have the disease, what is the probability a person has the disease given that the test result is positive?
	\end{problem}

	\begin{problem}
	A total of 46\% of the voters in a certain city classify themselves as Independents, whereas 30\% classify themselves as Liberals and 24\% as Conservative. In a recent local election, 35\% of the Independents, 62\% of the Liberals, and 58\% of the Conservatives voted. A voter is chosen at random. Given that this person voted in the local election, what is the probability that the person is a) an Independent? b) a Liberal? c) a Conservative?
	\end{problem}
	
	\begin{problem}
	When a dice $x$ is tossed it lands on \epsdice{2} with probability $1/2$ and all the other outcomes are equally likely to happen. When a dice $y$ is tossed, it lands on \epsdice{3} with probability $1/2$ and all the other outcomes are equally likely to happen. Suppose that one of these dice is randomly chosen and then tossed. What is the probability that dice $x$ was tossed, if the die landed on \epsdice{2}?
	\end{problem}
	
	\subsection{Independent Events}
	
	\begin{problem}
	Three brands of coffee, $x$, $y$, and $z$, are to be ranked according to taste by a judge. Define the following events. $A$: ``Brand $x$ is preferred to $y$, $B$: ``Brand $x$ is ranked best'', $C$: ``Brand $x$ is ranked second best'' and $D$: ``Brand $x$ is ranked third best''. If the judge actually has no taste preference and randomly assigns ranks to the brands, is event $A$ independent of (a) event $B$? (b) event C? (c) event $D$?
	\end{problem}
	
	\begin{problem}
	Cards are dealt, one at a time, from a standard 52-card deck. If $A_i$ denotes the event ``the $i$-th card dealt is a spade''. Are $A_1$ and $A_2$ independent?
	\end{problem}	

	\begin{problem}
	A system composed of $5$ separate components is said to be a parallel system if it functions when at least one of the components functions. For such a system, if component $i$, independent of other components, functions with probability $p_i$, $i = 1, 2, \ldots , 5$, what is the probability that the system functions?
	\end{problem}
	
	\begin{problem}
	Let $(S , \mathcal{A} , P )$ be a probability space. Prove that 
		\begin{enumerate}[label=\alph*)]
		\item If $A$ and $B$ are independent events with $0 < P (A), P (B) < 1$, then $A$ and $\overline{B}$ are independent.
		\item If $A$ and $B$ are independent events with $0 < P (A) , P (B) < 1$, then $\overline{A}$ and $\overline{B}$ are independent.
		\end{enumerate}
	\end{problem}

\end{document}