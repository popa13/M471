\documentclass[12pt]{article}
\usepackage[utf8]{inputenc}

\usepackage{lmodern}

\usepackage{enumitem}
\usepackage[margin=2cm]{geometry}

\usepackage{amsmath, amsfonts, amssymb}
\usepackage{graphicx}
\usepackage{subfigure}
\usepackage{tikz}
\usepackage{pgfplots}
\usepackage{multicol}

\usepackage{titlesec}
\usepackage{environ}
\usepackage{xcolor}
\usepackage{fancyhdr}
\usepackage[colorlinks = true, linkcolor = black]{hyperref}
\usepackage{xparse}
\usepackage{enumitem}
\usepackage{comment}
\usepackage{wrapfig}
\usepackage[capitalise]{cleveref}
\usepackage{epsdice}
\usepackage{circledsteps}

\usepackage{url}
\usepackage{calc}
%\usepackage{subcaption}
\usepackage[indent=0pt]{parskip}

\usepackage{array}
\usepackage{blkarray,booktabs, bigstrut}
\usepackage{bigints}

\pgfplotsset{compat=1.16}

% MATH commands
\newcommand{\ga}{\left\langle}
\newcommand{\da}{\right\rangle}
\newcommand{\oa}{\left\lbrace}
\newcommand{\fa}{\right\rbrace}
\newcommand{\oc}{\left[}
\newcommand{\fc}{\right]}
\newcommand{\op}{\left(}
\newcommand{\fp}{\right)}

\newcommand{\bi}{\mathbf{i}}
\newcommand{\bj}{\mathbf{j}}
\newcommand{\bk}{\mathbf{k}}
\newcommand{\bF}{\mathbf{F}}

\newcommand{\mR}{\mathbb{R}}
\newcommand{\mC}{\mathbb{C}}
\newcommand{\mT}{\mathbb{T}}
\newcommand{\mD}{\mathbb{D}}

\newcommand{\ra}{\rightarrow}
\newcommand{\Ra}{\Rightarrow}

\newcommand{\sech}{\mathrm{sech}\,}
\newcommand{\csch}{\mathrm{csch}\,}
\newcommand{\curl}{\mathrm{curl}\,}
\newcommand{\dive}{\mathrm{div}\,}

\newcommand{\ve}{\varepsilon}
\newcommand{\spc}{\vspace*{0.5cm}}

\DeclareMathOperator{\Ran}{Ran}
\DeclareMathOperator{\Dom}{Dom}
\DeclareMathOperator{\re}{Re}
\DeclareMathOperator{\im}{Im}
%\DeclareMathOperator{\arg}{arg}

\usepackage{pifont}

\newcommand{\club}{\ding{168}}
\newcommand{\spade}{\ding{171}}
\newcommand{\ddiamond}{{\color{red}\ding{169}}}
\newcommand{\heart}{{\color{red}\ding{170}}}

% Playing card
\newcommand{\cardH}[2]{%
\begin{tikzpicture}[trim right = #2pt, scale=0.1]
  % Card outline
  \draw[thick, red] (0,0) rectangle (5,3.5);
  % Card suit and value
  \node at (1.4,1.75) {\tiny\color{red} #1};
  \node[red] at (3.5, 1.75) {\scriptsize\heart};
\end{tikzpicture}
}
\newcommand{\cardD}[2]{%
\begin{tikzpicture}[trim right = #2pt, scale=0.1]
  % Card outline
  \draw[thick, red] (0,0) rectangle (5,3.5);
  % Card suit and value
  \node at (1,1.75) {\tiny\color{red} #1};
  \node[red] at (3.5, 1.75) {\footnotesize\ddiamond};
\end{tikzpicture}
}
\newcommand{\cardS}[2]{%
\begin{tikzpicture}[trim right = #2pt, scale=0.1]
  % Card outline
  \draw[thick, black] (0,0) rectangle (5,3.5);
  % Card suit and value
  \node at (1,1.75) {\tiny\color{black} #1};
  \node[black] at (3.5, 1.75) {\footnotesize\spade};
\end{tikzpicture}
}
\newcommand{\cardC}[2]{%
\begin{tikzpicture}[trim right = #2pt, scale=0.1]
  % Card outline
  \draw[thick, black] (0,0) rectangle (5,3.5);
  % Card suit and value
  \node at (1,1.75) {\tiny\color{black} #1};
  \node[black] at (3.5, 1.75) {\footnotesize\club};
\end{tikzpicture}
}

%% Defining example environment
\newcounter{totNumProblems}
\newcounter{problem}[section]
\NewEnviron{problem}%
	{%
	\noindent\refstepcounter{problem}\refstepcounter{totNumProblems}\fcolorbox{gray!40}{gray!40}{\textsc{\textcolor{black}{Problem~\theproblem.}}}%
	%\fcolorbox{black}{white}%
		{  %\parbox{0.95\textwidth}%
			{
			\BODY
			}%
		}%
	}


%% Redefining sections
\newcommand{\sectionformat}[1]{%
    \begin{tikzpicture}[baseline=(title.base)]
        \node[rectangle, draw] (title) {#1 \thesection};
    \end{tikzpicture}
    
    \noindent\hrulefill
}

\renewcommand{\thesection}{\Alph{section}}
\renewcommand{\thesubsection}{\Alph{section}.\arabic{subsection}}

% default values copied from titlesec documentation page 23
% parameters of \titleformat command are explained on page 4
\titleformat{\section}{\centering\normalfont\large\scshape}{}{1em}{\centering\sectionformat}

%% Set counters for sections to none
%\setcounter{secnumdepth}{0}

%% Set the footer/headers
\pagestyle{fancy}
\fancyhf{}
\renewcommand{\headrulewidth}{0pt}
\renewcommand{\footrulewidth}{2pt}
\lfoot{P.-O. Paris{\'e}}
\cfoot{MATH 471}
\rfoot{Page \thepage}

\begin{document}
\hrulefill

\begin{minipage}{0.33\textwidth}
\textsc{Math 471}
\end{minipage} \hfill 
\begin{minipage}{0.32\textwidth}
\centering
\textsc{Solutions to Problems} \\
Chapter C
\end{minipage}
 \hfill 
 \begin{minipage}{0.33\textwidth}
 \flushright \textsc{Fall 2023}
 \end{minipage}

\hrulefill

\setcounter{section}{3}

\subsection{Discrete Random Variables}
    
    \begin{problem}
    The $\im Z$ is discrete because $\im X$ and $\im Y$ are discrete sets. 

    Let $z \in \mR$. If $\{ Z = z \} = \varnothing$, then $\{ Z = z \}$ is an event because the set $\varnothing$ is always an event. Assume that $\{ Z = z \} \neq \varnothing$. We have to consider two cases.
        \begin{enumerate}[label=\roman*)]
            \item \underline{$z = 0$.} In this case, the only way that $Z (s) = 0$ is if $X (s) = 0$ or $Y (s) = 0$. Therefore,
                \begin{align*}
                \{ Z = 0 \} = \{ X = 0 \} \cup \{ Y = 0 \} .
                \end{align*} 
            Since $\{ X = 0 \}$ and $\{ Y = 0  \}$ are events, we conclude that $\{ Z = 0 \}$ are events (recall that, by assumption, $X$ and $Y$ are discrete random variables). 
            \item \underline{$z \neq 0$.} In this case, the functions $X$ and $Y$ can't take the value $0$. If $s \in \{ Z = z \}$, then $X(s) Y(s)= Z(s) = z$. Therefore $X(s) = z/ Y(s)$. Let $y = Y (s)$. Then $X(s) = z/y$ and $Y(s) = y$. In other words, $s \in \{ X = z/y\} \cap \{ Y = y \}$. On the other hand, if $s \in \{ X = z/y \} \cap \{ Y = y \}$, then $X(s) = z/y$ and $Y(s) = y$. Therefore, $Z(s) = X(s) Y(s) = (z/y) y = z$ and then $s \in \{ Z = z \}$. In summary, we have just proved that
            \begin{align*}
            \{ Z = z \} = \bigcup_{y \in \im Y, y \neq 0} \Big( \{ X = z /y \} \cap \{ Y = y \} \Big) .
            \end{align*} 
            For a given $z \in \mR$, $z \neq 0$ and $y \in \im Y$, the event $\{ X = z / y \} \cap \{ Y = y \}$ is an event because $X$ and $Y$ are discrete random variables and $\mathcal{A}$ is an event space. Thus, a countable union of these events will remain an event. Hence, $\{ Z = z \}$ is an event. 
        \end{enumerate}
        In each case, $\{ Z = z \}$ is an event. The map $Z$ satisfies condition (a) and (b) in Definition C.1 and therefore $Z$ is a discrete random variable. \hfill $\triangle$
    \end{problem}

    \begin{problem}
    We have $\im (1_A) = \{ 0, 1 \}$, which is a finite set (therefore discrete). 

    Let $x \in \mR$. We have three cases to consider.
    \begin{enumerate}
        \item \underline{$x = 0$.} In this case, $\{ 1_A = 0\} = \overline{A}$. Since $A$ is an event, we know that $\overline{A}$ is also an event. Therefore, $\{ 1_A = 0 \}$ is an event.
        \item \underline{$x= 1$.} In this case, $\{ 1_A = 1 \} = A$ and $A$ is an event. Hence, $\{ 1_A = 1 \}$ is an event.
        \item \underline{$x \neq 0$ and $x \neq 1$.} In this case, $\{ 1_A = x \} = \varnothing$ because there is no $s$ such that $1_A (x) = x$ (the only possible values are $0$ and $1$ for $1_A$). Since $\varnothing$ is an event, $\{ 1_A = x \}$ is an event. \hfill $\triangle$
    \end{enumerate}
    \end{problem}

    \begin{problem}
    \begin{enumerate}[label=\alph*)]
        \item Let $x \in \mR$. If $\{ X \leq x \} = \varnothing$, then $\{ X \leq x \}$ is an event. Assume that $\{ X \leq x \} \neq \varnothing$. Since $X$ is a discrete random variable, the set $\im X$ is discrete. This means there are only a countable values of $\im X$ that can be smaller than the number $x$. List them in decreasing order, say $x_1$, $x_2$, $x_3$, $\ldots$, with $x_i \geq x_j$, when $i \leq j$ and $x_j \leq x$ for any $j$. Therefore, we can write
            \begin{align*}
            \{ X \leq x \} = \bigcup_{j = 1}^\infty \{ X = x_j \} .
            \end{align*} 
        The map $X$ is a discrete random variable. Therefore, each set $\{ X = x_j \}$ is an event and this implies that $\cup_{j = 1}^\infty \{ X = x_j \}$ is an event. Hence, $\{ X \leq x \}$ is an event.
        \item Let $x \in \mR$. We can write
            \begin{align*}
            \{ X < x \} = \{ X \leq x \} \cap \overline{\{ X = x \} } .
            \end{align*} 
        In other words, the set $\{ X < x \}$ is the set of $s \in S$ that belong to $\{ X \leq x \}$ but are not in $\{ X = x \}$. From part a), the set $\{ X \leq x \}$ is an event and from the fact that $X$ is assumed to be a discrete random variable, $\{ X = x \}$ is an event. Therefore, $\{ X \leq x \} \cap \overline{\{ X = x \}}$ is an event and hence $\{ X < x \}$ is an event.
        \item We have
            \begin{align*}
            \{ X \geq x \} = \overline{\{ X < x \}} .
            \end{align*} 
        From part b), we know that $\{ X < x \}$ is an event, hence $\{ X \geq x \}$ is also an event.
        \item We have
            \begin{align*}
            \{ X > x \} = \overline{\{ X \leq x \}}.
            \end{align*} 
        From part c), we know that $\{ X \leq x \}$ is an event, hence $\{ X > x \}$ is also an event.
    \end{enumerate}
    \end{problem}

    \begin{problem}
    Assume that $X$ is a discrete random variable. Then $\im X$ is discrete and $\{ X = x \}$ is an event for every $x \in \mR$. From Problem 3, part a), the set $\{ X \leq x \}$ is an event. Therefore, conditions a) and b) in the statement are satisfied.

    Assume that the two conditions in the statement are satisfied. Then, in particular, $\im X$ is discrete. Also, for an $x \in \mR$, the set $\{ X > x \}$ is an event because it is the complement of the event $\{ X \leq x \}$. Also, for an $x \in \mR$, we have
        \begin{align*}
        \{ X < x \} = \bigcup_{j = 1}^\infty \Big\{ X \leq x - \frac{1}{j} \Big\} .
        \end{align*} 
    This is a countable unions of the events $\{ X \leq x - \frac{1}{j} \}$ and therefore $\{ X < x \}$ is an event. But also $\{ X \geq x \}$ is also an event because it is the complement of $\{ X < x \}$. Let $x \in \mR$. We can write
        \begin{align*}
        \{X = x \} = \{ X \leq x \} \cap \{ X \geq x \} ,
        \end{align*} 
    the intersection of two events! So $\{ X = x \}$ is also an event. Hence $X$ is a discrete random variable.
    \end{problem}

\subsection{Probability Mass Functions}

    \begin{problem}
    The probability measure $P$ on $S$ is given by
    \begin{align*}
    P (\{ r \}) = \frac{2}{5} , \, P (\{ b \}) = \frac{2}{5} , \quad P (\{ y \}) = \frac{1}{5} .
    \end{align*} 
    The function $X : S \ra \mR$ is given by $X (\{ r \}) = -10$, $X (\{ b \}) = 10$, and $X (\{ y \}) = 20$. Therefore, we have
        \begin{itemize}
            \item $p_X (-10) = P (X = -10) = P (\{ r \}) = \frac{2}{5}$.
            \item $p_X (10) = P (X = 10) = P (\{ b \}) = \frac{2}{5}$.
            \item $p_X (20) = P (X = 20) = P (\{ y \}) = \frac{1}{5}$.
            \item $p_X (x) = 0$, for $x \neq -10, 10, 20$.
        \end{itemize} 
    \end{problem}

    \begin{problem}
    A child may or may not identify properly the picture. Let $w_1$, $w_2$, $w_3$ be the words corresponding to the animal in picture $p_1$, $p_2$, $p_3$. A child will identify correctly a picture if the word $w_i$ is put under the picture $p_i$. Therefore, we can identify an outcome as an ordered list of three symbols from $\{ w_1, w_2 , w_3 \}$. For example, $w_1 w_2 w_3$ means that the child identified the animal in picture $p_1$ as $w_1$, in picture $p_2$ as $w_2$, and in picture $p_3$ as $w_3$. Therefore, the sample space is
        \begin{align*}
        S = \{ w_1w_2w_3, w_1w_3w_2, w_2w_1w_3, w_3 w_2 w_1, w_3 w_1 w_2, w_2 w_1 w_3 \} .
        \end{align*} 
    If we just keep the numbers
        \begin{align*}
        S = \{ 123, 132, 213, 321, 312, 231 \} .
        \end{align*} 
    Each outcome are equally likely to happen, so with $1/6$ chance.

    Let $Y : S \ra \mR$. Notice that, if the child successfully matches 2 pictures with their words, then the third picture will be also successfully matched. Therefore, the child may correctly identify $0$, $1$, or $3$ of the pictures presented and $\im Y = \{ 0 , 1, 3 \}$. Then, we have $p_Y (y) = 0$ for any $y \neq 0, 1, 3$. For the other values of $y$:
        \begin{itemize}
            \item $p_Y (0) = P (Y = 0) = P (\{ 312, 231 \}) = 1/3$.
            \item $p_Y (1) = P (Y = 1) = P (\{ 132, 213, 312 \}) = \frac{1}{2}$.
            \item $p_Y (3) = P (Y = 3) = P (\{ 123 \}) = \frac{1}{6}$. 
        \end{itemize}
    \end{problem}

    \begin{problem}
    The sample space is all distinct subsets of two numbers from $\{ 1, 2, 3, 4, 5 \}$. There are $\binom{5}{2} = 10$ possible outcomes and all of the outcome are equally likely to occur. 
    \begin{enumerate}[label=\alph*)]
        \item We have $\im X = \{ 2, 3, 4, 5 \}$. The number $1$ is missing because in the two balls selected, if ball \#1 is selected, then the other ball's number is automatically one of $2$, $3$, $4$, $5$. We will present the pmf of $X$ in a table.
        \begin{center}
            \begin{tabular}{c|c|c|c|c}
            $x$ & 2 & 3 & 4 & 5 \\\hline
            $p_X (x)$ & $1/10$ & $1/5$ & $3/10$ & $2/5$
            \end{tabular}
        \end{center}  
    To compute $p_X (2)$, we first notice that $\{ X = 2 \} = \{ \{ 1, 2 \} \}$ and therefore $P (X = 2) = 1/10$. To compute $p_X (3)$, we first notice that $\{ X = 3 \} = \{ \{ 1 , 3 \} , \{ 2 , 3 \} \}$ and therefore $P (X = 3) = 2/10 = 1/5$. Similar calculations lead to the values of $p_X (4)$ and $p_X (5)$. Notice that $1/10 + 1/5 + 3/10 + 2/5 = 1$. 
        \item Removing the parenthesis in the set and considering them as unordered list, the outcomes of $S$ can be explicitly enumerated:
            \begin{align*}
            S = \{ 12, 13, 14, 15, 23, 24, 25, 34, 35, 45 \} .
            \end{align*} 
        Therefore, considering all the outcomes and adding the numbers, we see that $\im X = \{ 3, 4, 5, 6, 7, 8, 9 \}$. We have, more precisely, $X(12) = 3$, $X(13) = 4$, $X(14) = X(23) = 5$, $X(24) = X(15) = 6$, $X(25) = X(34) = 7$, $X(35) = 8$, $X(45) = 9$.
        Using the same strategy as in a), we find the following values for $p_X$.
        \begin{center}
            \begin{tabular}{c|c|c|c|c|c|c|c}
            $x$ & 3 & 4 & 5 & 6 & 7 & 8 & 9 \\ \hline
            $p_X (x)$ & $1/10$ & $1/10$ & $1/5$ & $1/5$ & $1/5$ & $1/10$ & $1/10$
            \end{tabular}
        \end{center}
    \end{enumerate}
    \end{problem}

    \begin{problem}
    If $p$ is a probability mass function, then we know it should satisfy $\sum_{k = 1}^\infty p(k) = 1$. This gives the following condition:
        \begin{align*}
        \sum_{k = 1}^\infty \frac{c}{k (k + 1)} = 1 .
        \end{align*} 
    Now, using the trick $\frac{1}{k (k + 1)} = \frac{1}{k} - \frac{1}{k + 1}$, we see that the series $\sum_{k = 1}^\infty  \frac{1}{k (k + 1)}$ is convergent and
        \begin{align*}
        \sum_{k = 1}^\infty \frac{1}{k (k + 1)} = \lim_{N \ra \infty} \sum_{k = 1}^N \Big( \frac{1}{k} - \frac{1}{k + 1} \Big) = \lim_{N \ra \infty} 1 - \frac{1}{N + 1} = 1 .
        \end{align*} 
    Therefore, using the properties of series, we see that
    \begin{align*}
    \sum_{k = 1}^\infty \frac{c}{k (k + 1)} = 1 \iff c \sum_{k = 1}^\infty \frac{1}{k (k + 1)} = 1 \iff c \cdot 1 = 1 \iff c = 1 .
    \end{align*} 
    This means the function $p$ is a pmf if and only if $c = 1$. 
    \end{problem}
    
\subsection{Functions of Discrete Random Variables}
    
    \begin{problem}
    Setting $X = 1$, $X = 2$, $X = 3$, and $X = 4$ in the expression of $Y$, we get $Y = 0, 3, 8, 15$. Therefore, $\im Y = \{ 0, 3, 8, 15 \}$.

    Using Theorem 3, with $g(x) = x^2 - 1$, we have
        \begin{align*}
        p_Y (y) = \sum_{x \in g^{-1} (y)} P (X = x ) .
        \end{align*} 
    For 
        \begin{itemize}
            \item $y = 0$, we have $g^{-1} (0) = \{ x \in \im X \, : \, g(x) = 0 \} = \{ 1 \}$;
            \item $y = 1$, we have $g^{-1} (3) = \{ x \in \im X \, : \, g(x) = 3 \} = \{ 2 \}$;
            \item $y = 8$, we have $g^{-1} (8) = \{ x \in \im X \, : \, g (x) = 8 \} = \{ 3 \}$;
            \item $y = 15$, we have $g^{-1} (15) = \{ x \in \im X \, : \, g (x) = 15 \} = \{ 4 \}$. 
        \end{itemize}
    Therefore,
        \begin{itemize}
            \item $p_Y (0) = P (X = 1) = 0.4$.
            \item $p_Y (3) = P (X = 2) = 0.3$.
            \item $p_Y (8) = P (X = 3) = 0.2$.
            \item $p_Y (15) = P (X = 4) = 0.1$.
            \item $p_Y (y) = 0$ for any other values $y$ different from $0, 3, 8, 15$. \hfill $\triangle$
        \end{itemize}
    \end{problem}

    \begin{problem}
    Setting $X = 1, 2, 3, 4$ in the expression of $Y$, we get $Y = 1, 0, -1, 0$ respectively. Therefore, $\im Y = \{ -1, 0, 1 \}$.

    Using Theorem 3 again, but with $g (x) = \sin (\frac{\pi}{2} x )$, we have
        \begin{align*}
        p_Y (y) = \sum_{x \in g^{-1} (y)} P (X = x ) .
        \end{align*} 
    For
        \begin{itemize}
            \item $y = -1$, $g^{-1}(-1) = \{ 3 \}$;
            \item $y = 0$, $g^{-1} (0) = \{ 2, 4 \}$;
            \item $y = 1$, $g^{-1} (1) = \{ 1 \}$.
        \end{itemize}
    Therefore,
        \begin{itemize}
            \item $p_Y (-1) = P (X = 3) = 0.2$.
            \item $p_Y (0) = P (X = 2) + P (X = 4) = 0.3 + 0.1 = 0.4$.
            \item $p_Y (1) = P (X = 1) = 0.4$. \hfill $\triangle$
        \end{itemize}
    \end{problem}

\subsection{Expectation and Variance}

\begin{problem}
Let $t_1$ be the label ``the dimensions of the trailer are $8 \times 10 \times 30$''. and let $t_2$ be the label ``the dimensions of the trailer are $8 \times 10 \times 40$''. Given a trailer, the possible outcome is a trailer of type $t_1$ or of type $t_2$. Therefore, $S = \{ t_1 , t_2 \}$ with $P (\{ t_1 \}) = 0.3$ and $P (\{ t_2 \}) = 0.7$.

Let $X$ be the map giving the volume of a trailer. We have 
$$
    X (t_1) = 8 \cdot 10 \cdot 30 = 2400 \quad \text{ and } \quad X (t_2) = 8 \cdot 10 \cdot 40 = 3200 .
$$
Therefore, we get
\[
    \mathrm{Exp} (X) = X( t_1) P (X = 2400) + X (t_2) P (X = 3200) = (2400) (0.3) + (3200)(0.7) = 2960 .
\]
The average volume shipped per trailer load is $2960 \mathrm{ft}^3$. \hfill $\triangle$
\end{problem}

\begin{problem}
A firm can be assigned one or two contracts. Therefore, we can generate the set of outcomes as couple of letters taken from $\{ a, b, c \}$. For example $aa$ means $a$ was assigned to the two contracts, but $ab$ or $ba$ means that $a$ and $b$ was assigned to one of the contracts. The sample space $S$ is
    \begin{align*}
    S = \{ aa, ab, ba, ac, ca, bb, bc, cb, cc \} .
    \end{align*}
Since the firms are assigned a contract at random, each outcome are equally likely to occur, so with $1/9$. 

\begin{enumerate}[label=\alph*)]
    \item In the first scenario, assume that $X$ is the possible profit made by firm $A$ after the contracts were assigned. Therefore, this means
        \begin{itemize}
            \item $X (aa) = 180,000$.
            \item $X(ab) = X (ba) = X(ac) = X (ca) = 90,000$.
            \item $X(bb) = X(bc) = X(cb) = X (cc) = 0$.
        \end{itemize}
    The expectation is then calculated as followed:
    \begin{align*}
        \mathrm{Exp} (X) &= 180,000 P (X = 180,000) + 90,000 P (X = 90,000) + 0 P (X = 0) \\ 
        &= 180,000 P (\{ aa \}) + 90,000 (P (\{ ab , ba, ac, ca \}) + 0 \\ 
        &= \frac{180,000}{9} + \frac{90,000 \cdot 4}{9} \\ 
        &=  20,000 + 40,000 \\
        &= 60,000
    \end{align*}
    \item Let $Y$ be the profit made by firms $A$ and $B$ after the contrasts were assigned. Therefore, this means
        \begin{itemize}
            \item $X (aa) = X(bb) = X(ab) = X (ba) = 180,000$.
            \item $X (ac) = X(ca) = X(bc) = X(cb) = 90,000$.
            \item $X (cc) = 0$.
        \end{itemize}
    The expectation is then calculated as followed:
        \begin{align*}
        \mathrm{Exp} (X) &= 180,000 P (X = 180,000) + 90,000 P (X = 90,000) + 0 P (X = 0) \\ 
        &= 180,000 P (\{ aa, bb, ab, va \}) + 90,000 P (\{ ac, ca, bc, cb \}) \\ 
        &= \frac{(180,000)(4)}{9} + \frac{(90,000)(4)}{9} \\ 
        &= 80,000 + 40,000 \\ 
        &= 120,000 . \tag*{$\triangle$}
        \end{align*} 
\end{enumerate}
\end{problem}

\begin{problem}
We have $\im X = \{ 1, 2, 3, 4, 5, 6 \}$ and each value of $X$ has a chance of $1/6$ to occur. Therefore,
\[
    \mathrm{Exp} (X) = \frac{1}{6} + \frac{2}{6} + \frac{3}{6} + \frac{4}{6} + \frac{5}{6} + \frac{6}{6} = \frac{21}{6}  = 3 \tfrac{1}{2} .
\]

The variance is calculated using formula in the Theorem C.6. We first have $\im X^2 = \{ 1 , 4, 9, 16, 25, 36 \}$ and
\[
    \mathrm{Exp} (X^2) = \frac{1}{6} + \frac{4}{6} + \frac{9}{6} + \frac{16}{6} + \frac{25}{6} + \frac{36}{6} =  \frac{91}{6} = 15 \tfrac{1}{6} . 
\]
Therefore,
\[
    \mathrm{Var} (X) = \mathrm{Exp} (X^2) - (\mathrm{Exp} (X))^2 = \frac{91}{6} - \frac{21}{6} = \frac{70}{6} = 11 \tfrac{2}{3} .
\]
Thus,
\[
    \sigma = \sqrt{\mathrm{Var} (X)} = \sqrt{70/6} \approx 3.4157 . \tag*{$\triangle$}
\]
\end{problem}

\begin{problem}
By the formula in Theorem C.6, we have
\[
    \mathrm{Var} (aX + b) = \mathrm{Exp} ( (aX + b)^2) - (\mathrm{Exp} (aX +b))^2 .
\]
We have $(aX + b)^2 = a^2 X^2 + 2abX + b^2$ and $\mathrm{Exp} (aX + b) = a \mathrm{Exp} (X) + b$. Therefore,
\begin{align*}
\mathrm{Var} (aX + b) &= a^2 \mathrm{Exp} (X^2) + 2 ab \mathrm{Exp} (X) + b^2 - a^2 (\mathrm{Exp} (X))^2 - 2ab \mathrm{Exp} (X) - b^2 \\ 
&= a^2 \mathrm{Exp} (X^2) - a^2 \mathrm{Exp} (X) \\ 
&= a^2 \mathrm{Var} (X) . \tag*{$\triangle$}
\end{align*}
\end{problem}

\subsection{Conditional Expectation and the Partition Theorem}

\begin{problem}
Let $B = \{ X = x \}$. Then, we have
    \[
        E (g(X) | B) = \sum_{y \in \im g (X)} y P (g (X) = y | B ) . 
    \]
However, if $B$ has occurred, then $X = x$ and the sum over $\im g (X)$ is restricted to the value $y = g (x)$. Hence,
    \[
        E (g (X) | B) = g (x) P (g (X) = g(x) | B) = g (x) \frac{ P (\{ g (X) = g (x) \} \cap \{ X = x \} )}{P (X = x)} .
    \]
But $\{ g (X) = g (x) \} = \{ X = x \}$ and therefore
    \[
        E (g (X) | B) = g (x) \frac{P (X = x)}{P (X = x)} = g (x) . \tag*{$\triangle$}
    \]
\end{problem}

\subsection{Examples of Discrete Random Variables}

\begin{problem}
The map $X$ has a discrete range, that is $\im X = \{ 0, 1, 2, 3, \ldots , 30 \}$. However it does not have a binomial distribution because the probability that there is rain on a given day varies from day to day. Therefore, the parameter $p$ is not fixed. \hfill $\triangle$
\end{problem}

\begin{problem}
\begin{enumerate}[label=\alph*)]
    \item In this case, if it was explicitly mentioned ``The number of students in a sample of X students who took the SAT", then we could model the distribution of $X$ on the binomial distribution with $n= 100$ and $q = 0.45$. Unfortunately, it is not mentioned and therefore we can't model the distribution with a binomial distribution.
    \item It can't be model by a binomial distribution because there is not enough information to find the parameter $q$. We will see later that the distribution of the scores of the 100 students can be model by a normal distribution.
    \item If $X_j$ is the random variable ``The student labeled $j$ scored above average on the SAT'', then $X$: ``the number of students in the sample who scored above average on the SAT'', which is equal to $X_1 + X_2 + \ldots + X_{100}$, has a binomial distribution. In this case, $n = 100$ and the value of $q$ is not possible to find. We would need more information to compute an approximate value for $q$. For example, with the additional assumption that the distribution of the student's scores is a Normal distribution, then we can assume that $q = 0.5$, because $P (X > \mu ) = 0.5$ for any normal distribution.  
    \item It can't be model by a binomial distribution because there is not enough information to find the parameter $q$. We would need additional information on the average time of a student to complete the test. We will see later that the distribution of the random variable will be modeled by a Normal distribution. \hfill $\triangle$
\end{enumerate}
\end{problem}

\begin{problem}
By Definition C.3, we have
    \[
        \mathrm{Exp} (X) = \sum_{k = 0}^n k P (X = k) = \sum_{k = 0}^n k \frac{n!}{k! (n - k)!} q^k (1 - q)^{n - k} . 
    \]
The term with $k = 0$ disappears and we enter into the following chain of equalities:
    \begin{align*}
    \mathrm{Exp} (X) &= \sum_{k= 1}^n \frac{k n!}{k! (n - k)!} q^k (1 - q)^{n - k} \\ 
    &= \sum_{k = 0}^{n - 1} \frac{(k + 1) n!}{(k + 1)! (n - k - 1)!} q^{k + 1} (1 - q)^{n - k - 1} \\ 
    &= n q \sum_{k = 0}^{n-1} \frac{(n - 1)!}{k! (n - 1 - k)!} q^k (1 - q)^{n - k - 1} \\ 
    &= nq (q + (1 - q))^{n - 1} \\ 
    &= nq .
    \end{align*} 

To compute the $\mathrm{Var} (X)$, we use the formula in Theorem C.6. We have $\mathrm{Exp} (X) = n q$ from the previous calculations. We need to compute $\mathrm{Exp} (X^2)$. By the Theorem C.4 we have
\begin{align*}
\mathrm{Exp} (X^2) = \sum_{k = 0}^n k^2 P (X^2 = k^2) = \sum_{k = 0}^n k^2 P (X = k) ,
\end{align*} 
where $\{ X^2 = k^2 \} = \{ X = k \}$ because $X$ assumes only non-negative integer values. Therefore,
\begin{align*}
\mathrm{Exp} (X^2) &= \sum_{k = 0}^n \frac{k^2 n!}{k! (n - k)!} q^k (1 - q)^{n - k} \\ 
&= \sum_{k = 1}^n \frac{k^2 n!}{k! (n - k)!} q^k (1 - q)^{n - k} \\ 
&= nq \sum_{k = 0}^{n-1} \frac{(k+1) (n-1)!}{k! (n-1-k)!} q^k (1 - q)^{n-1-k} \\ 
&= nq \Big( \sum_{k = 0}^{n-1} \frac{k (n -1)!}{k! (n - 1-k)!} q^k (1 - q)^{n-1-k} + \sum_{k = 0}^{n-1} \frac{(n-1)!}{k! (n - 1 - k)!} q^k (1 - q)^{n-1-k} \Big) \\ 
&= nq \Big( \sum_{k = 1}^{n-1} \frac{k (n -1)!}{k! (n - 1 - k)!} q^k (1-q)^{n - 1 - k} + 1 \Big) \\ 
&= nq \Big( (n - 1)q \sum_{k = 0}^{n - 2} \frac{(n - 2)!}{k! (n - 2 - k)!} q^k (1 - q)^{n - 2 - k} + 1 \Big) \\ 
&= nq \Big( (n - 1)q + 1 ) \\ 
&= nq (nq - q + 1) \\ 
&= n^2 q^2 + nq (1 - q) .
\end{align*} 
Hence,
\begin{align*}
\mathrm{Var} (X) = n^2 q^2 + nq (1 - q) - n^2 q^2 = nq (1 - q) . \tag*{$\triangle$}
\end{align*} 
\end{problem}

\end{document}