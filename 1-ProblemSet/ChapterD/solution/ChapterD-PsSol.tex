\documentclass[12pt]{article}
\usepackage[utf8]{inputenc}

\usepackage{lmodern}

\usepackage{enumitem}
\usepackage[margin=2cm]{geometry}

\usepackage{amsmath, amsfonts, amssymb}
\usepackage{graphicx}
\usepackage{subfigure}
\usepackage{tikz}
\usepackage{pgfplots}
\usepackage{multicol}

\usepackage{titlesec}
\usepackage{environ}
\usepackage{xcolor}
\usepackage{fancyhdr}
\usepackage[colorlinks = true, linkcolor = black]{hyperref}
\usepackage{xparse}
\usepackage{enumitem}
\usepackage{comment}
\usepackage{wrapfig}
\usepackage[capitalise]{cleveref}
\usepackage{epsdice}
\usepackage{circledsteps}

\usepackage{url}
\usepackage{calc}
%\usepackage{subcaption}
\usepackage[indent=0pt]{parskip}

\usepackage{array}
\usepackage{blkarray,booktabs, bigstrut}
\usepackage{bigints}

\pgfplotsset{compat=1.16}

% MATH commands
\newcommand{\ga}{\left\langle}
\newcommand{\da}{\right\rangle}
\newcommand{\oa}{\left\lbrace}
\newcommand{\fa}{\right\rbrace}
\newcommand{\oc}{\left[}
\newcommand{\fc}{\right]}
\newcommand{\op}{\left(}
\newcommand{\fp}{\right)}

\newcommand{\bi}{\mathbf{i}}
\newcommand{\bj}{\mathbf{j}}
\newcommand{\bk}{\mathbf{k}}
\newcommand{\bF}{\mathbf{F}}

\newcommand{\mR}{\mathbb{R}}
\newcommand{\mC}{\mathbb{C}}
\newcommand{\mT}{\mathbb{T}}
\newcommand{\mD}{\mathbb{D}}

\newcommand{\ra}{\rightarrow}
\newcommand{\Ra}{\Rightarrow}

\newcommand{\sech}{\mathrm{sech}\,}
\newcommand{\csch}{\mathrm{csch}\,}
\newcommand{\curl}{\mathrm{curl}\,}
\newcommand{\dive}{\mathrm{div}\,}

\newcommand{\ve}{\varepsilon}
\newcommand{\spc}{\vspace*{0.5cm}}

\DeclareMathOperator{\Ran}{Ran}
\DeclareMathOperator{\Dom}{Dom}
\DeclareMathOperator{\re}{Re}
\DeclareMathOperator{\im}{Im}
%\DeclareMathOperator{\arg}{arg}

\usepackage{pifont}

\newcommand{\club}{\ding{168}}
\newcommand{\spade}{\ding{171}}
\newcommand{\ddiamond}{{\color{red}\ding{169}}}
\newcommand{\heart}{{\color{red}\ding{170}}}

% Playing card
\newcommand{\cardH}[2]{%
\begin{tikzpicture}[trim right = #2pt, scale=0.1]
  % Card outline
  \draw[thick, red] (0,0) rectangle (5,3.5);
  % Card suit and value
  \node at (1.4,1.75) {\tiny\color{red} #1};
  \node[red] at (3.5, 1.75) {\scriptsize\heart};
\end{tikzpicture}
}
\newcommand{\cardD}[2]{%
\begin{tikzpicture}[trim right = #2pt, scale=0.1]
  % Card outline
  \draw[thick, red] (0,0) rectangle (5,3.5);
  % Card suit and value
  \node at (1,1.75) {\tiny\color{red} #1};
  \node[red] at (3.5, 1.75) {\footnotesize\ddiamond};
\end{tikzpicture}
}
\newcommand{\cardS}[2]{%
\begin{tikzpicture}[trim right = #2pt, scale=0.1]
  % Card outline
  \draw[thick, black] (0,0) rectangle (5,3.5);
  % Card suit and value
  \node at (1,1.75) {\tiny\color{black} #1};
  \node[black] at (3.5, 1.75) {\footnotesize\spade};
\end{tikzpicture}
}
\newcommand{\cardC}[2]{%
\begin{tikzpicture}[trim right = #2pt, scale=0.1]
  % Card outline
  \draw[thick, black] (0,0) rectangle (5,3.5);
  % Card suit and value
  \node at (1,1.75) {\tiny\color{black} #1};
  \node[black] at (3.5, 1.75) {\footnotesize\club};
\end{tikzpicture}
}

%% Defining example environment
\newcounter{totNumProblems}
\newcounter{problem}[section]
\NewEnviron{problem}%
	{%
	\noindent\refstepcounter{problem}\refstepcounter{totNumProblems}\fcolorbox{gray!40}{gray!40}{\textsc{\textcolor{black}{Problem~\theproblem.}}}%
	%\fcolorbox{black}{white}%
		{  %\parbox{0.95\textwidth}%
			{
			\BODY
			}%
		}%
	}


%% Redefining sections
\newcommand{\sectionformat}[1]{%
    \begin{tikzpicture}[baseline=(title.base)]
        \node[rectangle, draw] (title) {#1 \thesection};
    \end{tikzpicture}
    
    \noindent\hrulefill
}

\renewcommand{\thesection}{\Alph{section}}
\renewcommand{\thesubsection}{\Alph{section}.\arabic{subsection}}

% default values copied from titlesec documentation page 23
% parameters of \titleformat command are explained on page 4
\titleformat{\section}{\centering\normalfont\large\scshape}{}{1em}{\centering\sectionformat}

%% Set counters for sections to none
%\setcounter{secnumdepth}{0}

%% Set the footer/headers
\pagestyle{fancy}
\fancyhf{}
\renewcommand{\headrulewidth}{0pt}
\renewcommand{\footrulewidth}{2pt}
\lfoot{P.-O. Paris{\'e}}
\cfoot{MATH 471}
\rfoot{Page \thepage}

\begin{document}
\hrulefill

\begin{minipage}{0.33\textwidth}
\textsc{Math 471}
\end{minipage} \hfill 
\begin{minipage}{0.32\textwidth}
\centering
\textsc{Problems Set} \\
Chapter D
\end{minipage}
 \hfill 
 \begin{minipage}{0.33\textwidth}
 \flushright \textsc{Fall 2023}
 \end{minipage}

\hrulefill

\setcounter{section}{4}

\subsection{Distribution Function}

\begin{problem}

Let $\alpha$ be a number between $0$ and $1$. If $x_1 \leq x_2$, then
    \[
        F(x_1) = \alpha F_1(x_1) + (1 - \alpha ) F_2 (x_2) \leq \alpha F_1 (x_2) + (1 - \alpha ) F_2 (x) = F (x_2) .
    \]
Therefore, $F$ is inscreasing. We also have
    \[
        \lim_{x \ra \infty} F(x) = \lim_{x \ra \infty} \alpha F_1 (x) + (1 - \alpha ) F_2 (x) = \alpha \lim_{x \ra \infty} F_1 (x) + (1 - \alpha ) \lim_{x \ra \infty} F_2 (x) = \alpha (1) + (1 - \alpha ) (1) = 1 .
    \]
Therefore, $F$ is distribution function. \hfill $\square$
\end{problem}

\begin{problem}
The values of $Y$ are always positive, so if $y < 0$, then $F_Y(y) = 0$. If $y \geq 0$, then $Y = X$ and therefore $F_Y (y) = F_X (y)$. Therefore, $F_Y = \max \{ 0 , F_X \}$. 
\end{problem}

\begin{problem}
The total integral should be $1$. We have
    \[
        \int_{-\infty}^\infty e^{-|t|} \, dt = 2
    \]
so that
    \[
        2c = 1 \quad \Rightarrow \quad c = 1/2 . \tag*{$\square$}
    \]
\end{problem}

\subsection{Continuous Random Variable}

\begin{problem}
For $x < 0$, we can differentiate and get
    \[
        f_X (x) = \frac{d}{dx} F_X (x) = \frac{d}{dx} \Big( \frac{1}{2 (1 + x^2 ) }\Big) = \frac{-x}{(1 + x^2)^2} .
    \]
For $x > 0$, we can differentiate and get
    \[
        f_X (x) = \frac{d}{dx} F_X (x) = \frac{d}{dx} \Big( \frac{1 + 2x^2}{2 (1 + x^2)} \Big) = \frac{x}{(1 + x^2)^2} .
    \]
At $x = 0$, $f_X(0)$ is not defined. \hfill $\square$
\end{problem}

\begin{problem}\label{Prob:FindDistroFunction}
When $x < -1$, then $F_X (x) = 0$ because $f_X (x) = 0$. When $-1 \leq x < 1$, then
    \[
        F_X (x) = \int_{-1}^x \frac{2}{\pi (1 + t^2)} \, dt = \frac{2}{\pi} \arctan (x) + 1/2
    \]
When $x \geq 1$, then
    \[
        F_X (x) = \frac{2}{\pi} \arctan (1) + \frac{1}{2} = 1 .
    \]
Hence
    \[
        F_X (x) = \left\lbrace \begin{matrix} 0 & x < -1 \\ 
        \frac{2}{\pi} \arctan (x) + \frac{1}{2} & -1 \leq x \leq 1 \\ 
        1 & x > 1 . \end{matrix} \right. \tag*{$\square$}
    \]
\end{problem}

\begin{problem}\label{Prob:FindDistroFunction2}
We must have $\lim_{x \ra \infty} F_X (x) = 1$, and so
    \[
        c\int_0^1 x (x - 1) \, dx = 1 \iff -\frac{c}{6} = 1 .
    \]
Hence we must have $c = -6$. \hfill $\square$
\end{problem}

\subsection{Functions of Random Variables}

\begin{problem}
When $Y = g (X)$ and $g$ is increasing, then we use the formula
    \[
        f_Y (y) = f_X (g^{-1} (y)) \frac{d}{dy} (g^{-1} (y)) .
    \]
Since $X$ has the exponential distribution,
    \[
        f_Y (y) = \lambda e^{-\lambda g^{-1} (y)} \frac{d}{dy} (g^{-1} (y)) .
    \]
When $g$ is decreasing, then
    \[
        f_Y (y) = - \lambda e^{-\lambda g^{-1} (y)} \frac{d}{dy} (g^{-1} (y)) .
    \]

    \begin{enumerate}[label=\alph*)]
        \item We have $g^{-1} (y) = (y - 5)/2$. Since $g$ is increasing, with $\frac{d}{dy} (g^{-1} (y)) = 1/2$, we obtain
            \[
                f_Y (y) = \frac{\lambda}{2} e^{-\frac{\lambda}{2} (y - 5)} .
            \]
        \item The inverse is found in the following way. Set $y = (1 + x)^{-1}$ and then
            \[
                y (1 + x) = 1 \iff y + yx = 1 \iff x = \frac{1 - y}{y} .
            \]
        Therefore $g^{-1} (y) = (1 - y)/y$. Since $g$ is decreasing, with $\frac{d}{dy} (g^{-1} (y)) = -1/(1-y)^2$, we obtain
            \[
                f_Y (y) = \frac{\lambda e^{-\frac{\lambda (1 - y)}{y}} (1 - y)}{y}. \tag*{$\square$}
            \]
    \end{enumerate}
\end{problem}

\begin{problem}
Since the range of $F$ is between $0$ and $1$, then $\im Y \in [0, 1]$ with each number between $0$ and $1$ being attained because of the continuity of $F$.

Since $F$ is an increasing function, we can apply the formula to find the density function of $Y$. For $y \in [0, 1]$, we have
    \[
        f_Y (y) = f_X (F^{-1} (y)) \frac{d}{dy} (F^{-1} (y)) .
    \]
Since $X$ is a continuous random variable, we have 
    \[
        F(x) = \int_{-\infty}^x f_X (t) \, dt .
    \]
A formula for the derivative of the inverse in terms of the initial function $F$ is
    \[
        \frac{d}{dy} (F^{-1} (y)) = \frac{1}{F' (F^{-1} (y))} .
    \]
Since $F' (x) = f_X (x)$, we therefore obtain
    \[
        f_Y (y) = \frac{f_X (F^{-1} (y))}{f_X (F^{-1} (y))} = 1 .
    \]
Therefore, the density function of $Y$ is $1$ identically on $[0, 1]$. Thus, $Y$ has a uniform distribution on $[0, 1]$. \hfill $\square$
\end{problem}

\begin{problem}
Set $g (x) = \frac{3x}{1 - x}$. Then, $g' (x) = \frac{1}{(1 - x)^2}$. The derivative is always positive, so $g$ is increasing. The density function of $Y$ is then given by the following formula:
    \[
        f_Y (y) = f_X (g^{-1} (y)) \frac{d}{dy} (g^{-1} (y)) .
    \]
After some calculations, we obtain
    \[
        g^{-1} (y) = \frac{y}{y + 3} \quad \text{ and } \quad \frac{d}{dy} (g^{-1} (y)) = \frac{3}{(3 + y)^2} .
    \]
Hence,
    \[
        f_Y (y) = f_X \Big( \frac{y}{y + 3} \Big) \frac{3}{(3 + y)^2} .
    \]
When $-3 < y < 0$, then $\frac{y}{y + 3} < 0$ because $y + 3 > 0$. Therefore $f_X (y/(y + 3)) = 0$. 

If $y \geq 0$, then $0 \leq \frac{y}{y + 3} \leq 1$, because $y \leq y + 3$ and $y + 3$ is positive. Therefore, $f_X (y/(y +3)) = 1$ and then
    \[
        f_Y (y) = \frac{3}{(3 + y)^2} ,
     \]
when $y \geq 0$.

If $y < -3$, then $y < y + 3$ implies that $1 > \frac{y + 3}{y}$ and therefore $\frac{y}{y + 3} > 1$. Therefore, $f_Y (y) = 0$. 

Hence, we get
    \[
        f_Y (y) = \left\lbrace \begin{matrix} 0 & y < 0 \\ 
        \frac{3}{(3 + y)^2} & y \geq 0 .
        \end{matrix} \right.
    \]

We can now find the distribution of $Y$. When $y < 0$, $F_Y (y)= 0$ because $f_Y (y) = 0$. When $y \geq 0$, then
    \[
        F_Y (y) = \int_0^y \frac{3}{(3 + t)^2} \, dt = \left. \Big( \frac{-3}{3 + t} \Big) \right|_0^y = 1 - \frac{3}{3 + y} .
    \]
Notice that $F_Y (y) = \frac{y}{y + 3} = g^{-1} (y)$, when $y \geq 0$. \hfill $\square$
\end{problem}

\subsection{Expectation of Continuous Random Variables}

\begin{problem}
The expectation is
    \begin{align*}
        \mathrm{Exp} (X) &= \int_{-\infty}^\infty x f_X (x) \, dx \\ 
        &= \int_{-1}^1 \frac{2x}{\pi (1 + x^2 )} \, dx \\ 
        &= 0
    \end{align*}
because the function $\frac{2x}{\pi (1 + x^2 )}$ is an odd function. The expected value of $X$ is therefore $0$. \hfill $\square$ 
\end{problem}

\begin{problem}
The expectation is
    \[
        \int_{-\infty}^\infty x f_X (x) \, dx = \int_0^1 -6 x^2 (x - 1) \, dx = \frac{1}{2} . 
    \]
The variance is
    \[
        \mathrm{Exp} (X^2) - (\mathrm{Exp} (X))^2 = \int_{-0}^1 -6 x^3 (x -1) \, \ dx - \frac{1}{4} = \frac{3}{10} - \frac{1}{4} = \frac{1}{20} .\tag*{$\square$}
    \]
\end{problem}

\subsection{Other Examples of Continuous Random Variables}

\begin{problem}
    \begin{enumerate}[label=\alph*)]
        \item We have $Z^2 < 1$ if and only if $-1 < Z < 1$. Therefore, 
            \[
                P (Z^2 < 1) = P (-1 < Z < 1 ) = P (Z < 1) - P (Z \leq -1) = P (Z \leq 1 ) - P (Z \leq -1).
            \]
        Using the table, 
            \[
                P (Z^2 < 1) = 0.84134 - 0.15866 = 0.68268 .
            \]
        \item Similar calculations:
            \[
                P (Z^2 < 3.84146) = P (Z < 1.96) - P (Z < -1.96) = 0.975 - 0.0025 = 0.950 . \tag*{$\square$}
            \]
    \end{enumerate}
\end{problem}

\begin{problem}
Define $Z = \frac{X - \mu}{0.3}$, so that $Z \sim N (0, 1)$. We want to know for which $\mu$, $P (X > 8) = 0.01$. Using $Z$, we want to find $\mu$ such that
    \[
        P \Big( Z > \frac{8 - \mu}{0.3} \Big) = 0.01  \iff P \Big( Z \leq \frac{8 - \mu}{0.3} \Big) = 0.99
    \]
The $z$-score corresponding to a probability of $0.1$ is $z = 2.325$. Therefore
    \[
        \frac{8 - \mu}{0.3} = 2.325 \iff \mu = 7.3025 . \tag*{$\square$}
    \]
\end{problem}

\begin{problem}
This is a little trick from Calculus IV. Let $I$ denote the integral we want to compute. Consider 
    \[
        I^2 = \Big( \int_{-\infty}^\infty e^{-x^2} dx \Big) \Big( \int_{-\infty}^\infty e^{-y^2} \, dy \Big) = \int_{-\infty}^\infty \int_{-\infty}^\infty e^{-x^2 - y^2} \, dy dx .
    \]
Let $R = \{ (x, y) \, : \, -\infty < x < \infty , -\infty < y < \infty \}$. Using polar coordinates, we see that $R = \{ (r, \theta ) \, : \, 0 \leq r < \infty , 0 \leq \theta \leq 2\pi \}$. Therefore,
    \[
        I^2 = \iint_R e^{-x^2 - y^2} \, dA = \int_0^{2\pi} \int_0^\infty r e^{-r^2} \, dr d\theta 
    \]
Using $u = r^2$, we see that
    \[
        \int_0^\infty re^{-r^2} \, dr = \frac{1}{2} \int_0^\infty e^{-u} \, du = \frac{1}{2} 
    \]
and hence
    \[
        I^2 = \pi \quad \Longrightarrow \quad I = \sqrt{\pi} . \tag*{$\square$}
    \]
\end{problem}

\begin{problem}
Here, we have $g (x) = e^{2x}$. Since it is an increasing function with $g^{-1} (y) = \frac{1}{2} \ln y$, using the formula, the density function of $Y$ is
    \[
        f_Y (y) = \frac{e^{-\frac{1}{4} (\ln (y))^2}}{2 \sqrt{2\pi} y} \quad (\text{for } y > 0) .
    \]
Therefore, the expected value is
    \[
        \mathrm{Exp} (Y) = \int_{-\infty}^\infty y f_Y (y) \, dy = \frac{1}{2 \sqrt{2\pi}} \int_0^\infty e^{-( \frac{\ln (y)}{2})^2} \, dy .
    \]
Let $u = \ln (y) / 2$, so that $du = \frac{1}{2y} \, dy$. This means
    \[
        \mathrm{Exp} (Y) = \frac{1}{\sqrt{2\pi}} \int_{-\infty}^\infty e^{-u^2}e^{2u} \, du = \frac{e}{\sqrt{2\pi}} \int_{-\infty}^\infty e^{- (u - 1)^2} \, du = e . \tag*{$\square$}
    \]
\end{problem}

\begin{problem}
If $X$ has the exponential distribution, then
    \[
        F_X (x) = \int_0^x \lambda e^{-\lambda t} \, dt = 1 - e^{-\lambda x} \quad (x > 0 ) .
    \]
Since $a, b > 0$, notice that
    \[
        \{ X > a + b \} \cap \{ X > a \} = \{ X > a + b \} .
    \]
Therefore, by definition of the conditional probability,
    \[
        P (X > a + b | X > a ) = \frac{P (X > a + b)}{P (X > a )} = \frac{1 - (1 - e^{-\lambda (a + b)})}{1 - (1 - e^{-\lambda a})} = e^{-\lambda b} = P (X > b ) . \tag*{$\square$}
    \]
\end{problem}



\end{document}