\documentclass[12pt]{article}
\usepackage[utf8]{inputenc}

\usepackage{lmodern}

\usepackage{enumitem}
\usepackage[margin=2cm]{geometry}

\usepackage{amsmath, amsfonts, amssymb}
\usepackage{graphicx}
\usepackage{subfigure}
\usepackage{tikz}
\usepackage{pgfplots}
\usepackage{multicol}

\usepackage{titlesec}
\usepackage{environ}
\usepackage{xcolor}
\usepackage{fancyhdr}
\usepackage[colorlinks = true, linkcolor = black]{hyperref}
\usepackage{xparse}
\usepackage{enumitem}
\usepackage{comment}
\usepackage{wrapfig}
\usepackage[capitalise]{cleveref}
\usepackage{epsdice}
\usepackage{circledsteps}

\usepackage{url}
\usepackage{calc}
%\usepackage{subcaption}
\usepackage[indent=0pt]{parskip}

\usepackage{array}
\usepackage{blkarray,booktabs, bigstrut}
\usepackage{bigints}

\pgfplotsset{compat=1.16}

% MATH commands
\newcommand{\ga}{\left\langle}
\newcommand{\da}{\right\rangle}
\newcommand{\oa}{\left\lbrace}
\newcommand{\fa}{\right\rbrace}
\newcommand{\oc}{\left[}
\newcommand{\fc}{\right]}
\newcommand{\op}{\left(}
\newcommand{\fp}{\right)}

\newcommand{\bi}{\mathbf{i}}
\newcommand{\bj}{\mathbf{j}}
\newcommand{\bk}{\mathbf{k}}
\newcommand{\bF}{\mathbf{F}}

\newcommand{\mR}{\mathbb{R}}
\newcommand{\mC}{\mathbb{C}}
\newcommand{\mT}{\mathbb{T}}
\newcommand{\mD}{\mathbb{D}}

\newcommand{\ra}{\rightarrow}
\newcommand{\Ra}{\Rightarrow}

\newcommand{\sech}{\mathrm{sech}\,}
\newcommand{\csch}{\mathrm{csch}\,}
\newcommand{\curl}{\mathrm{curl}\,}
\newcommand{\dive}{\mathrm{div}\,}

\newcommand{\ve}{\varepsilon}
\newcommand{\spc}{\vspace*{0.5cm}}

\DeclareMathOperator{\Ran}{Ran}
\DeclareMathOperator{\Dom}{Dom}
\DeclareMathOperator{\re}{Re}
\DeclareMathOperator{\im}{Im}
%\DeclareMathOperator{\arg}{arg}

\usepackage{pifont}

\newcommand{\club}{\ding{168}}
\newcommand{\spade}{\ding{171}}
\newcommand{\ddiamond}{{\color{red}\ding{169}}}
\newcommand{\heart}{{\color{red}\ding{170}}}

% Playing card
\newcommand{\cardH}[2]{%
\begin{tikzpicture}[trim right = #2pt, scale=0.1]
  % Card outline
  \draw[thick, red] (0,0) rectangle (5,3.5);
  % Card suit and value
  \node at (1.4,1.75) {\tiny\color{red} #1};
  \node[red] at (3.5, 1.75) {\scriptsize\heart};
\end{tikzpicture}
}
\newcommand{\cardD}[2]{%
\begin{tikzpicture}[trim right = #2pt, scale=0.1]
  % Card outline
  \draw[thick, red] (0,0) rectangle (5,3.5);
  % Card suit and value
  \node at (1,1.75) {\tiny\color{red} #1};
  \node[red] at (3.5, 1.75) {\footnotesize\ddiamond};
\end{tikzpicture}
}
\newcommand{\cardS}[2]{%
\begin{tikzpicture}[trim right = #2pt, scale=0.1]
  % Card outline
  \draw[thick, black] (0,0) rectangle (5,3.5);
  % Card suit and value
  \node at (1,1.75) {\tiny\color{black} #1};
  \node[black] at (3.5, 1.75) {\footnotesize\spade};
\end{tikzpicture}
}
\newcommand{\cardC}[2]{%
\begin{tikzpicture}[trim right = #2pt, scale=0.1]
  % Card outline
  \draw[thick, black] (0,0) rectangle (5,3.5);
  % Card suit and value
  \node at (1,1.75) {\tiny\color{black} #1};
  \node[black] at (3.5, 1.75) {\footnotesize\club};
\end{tikzpicture}
}

%% Defining example environment
\newcounter{totNumProblems}
\newcounter{problem}[section]
\NewEnviron{problem}%
	{%
	\noindent\refstepcounter{problem}\refstepcounter{totNumProblems}\fcolorbox{gray!40}{gray!40}{\textsc{\textcolor{black}{Problem~\theproblem.}}}%
	%\fcolorbox{black}{white}%
		{  %\parbox{0.95\textwidth}%
			{
			\BODY
			}%
		}%
	}


%% Redefining sections
\newcommand{\sectionformat}[1]{%
    \begin{tikzpicture}[baseline=(title.base)]
        \node[rectangle, draw] (title) {#1 \thesection};
    \end{tikzpicture}
    
    \noindent\hrulefill
}

\renewcommand{\thesection}{\Alph{section}}
\renewcommand{\thesubsection}{\Alph{section}.\arabic{subsection}}

% default values copied from titlesec documentation page 23
% parameters of \titleformat command are explained on page 4
\titleformat{\section}{\centering\normalfont\large\scshape}{}{1em}{\centering\sectionformat}

%% Set counters for sections to none
%\setcounter{secnumdepth}{0}

%% Set the footer/headers
\pagestyle{fancy}
\fancyhf{}
\renewcommand{\headrulewidth}{0pt}
\renewcommand{\footrulewidth}{2pt}
\lfoot{P.-O. Paris{\'e}}
\cfoot{MATH 471}
\rfoot{Page \thepage}

\begin{document}
\hrulefill

\begin{minipage}{0.33\textwidth}
\textsc{Math 471}
\end{minipage} \hfill 
\begin{minipage}{0.32\textwidth}
\centering
\textsc{Problems Set} \\
Chapter E
\end{minipage}
 \hfill 
 \begin{minipage}{0.33\textwidth}
 \flushright \textsc{Fall 2023}
 \end{minipage}

\hrulefill

\setcounter{section}{5}

\subsection{Bivariate Distributions}

\begin{problem}
Let $(X, Y)$ be a random vector with joint distribution $F_{X, Y}$. Prove that, for any $a < c$ and $b < d$,
    \[
        P (a < X \leq b , c < Y \leq d ) = F(b, d) + F(a, c) - F (a, d) - F(b, c) .
    \]
\end{problem}

\subsection{Continuous Random Vectors}

\begin{problem}
If $(X, Y)$ are continuous random vector with joint probability density function $f_{X, Y}$. Prove that
    \[
        P (a \leq X \leq b , c \leq Y \leq d ) = P (a < X < b , c < Y < d) .
    \]
\end{problem}

\begin{problem}
If a radioactive particle is randomly located in a square of unit length, a reasonable model for the joint density function for $X$ and $Y$ (the coordinates of the location of the radioactive particle) is
    \[
        f_{X, Y} (x, y) = \left\lbrace \begin{matrix} k x y & \text{if } (x, y) \in [0, 1] \times [0, 1] \\
        0 & \text{elsewhere} \end{matrix} \right. 
    \]
\begin{enumerate}[label=\alph*)]
\item Find the value $k$ that makes this a probability density function.
\item Find the joint distribution function for $X$ and $Y$.
\item Find $P (X \leq 0.5 , Y \leq 0.75)$.
\end{enumerate}
\end{problem}

\begin{problem}
Let $(X, Y)$ denote the coordinates of a point chosen at random inside a unit circle whose center is at the origin. Their joint probability density function is
    \[
        f_{X, Y} (x, y) = \left\lbrace \begin{matrix} 1/ \pi & \text{if } x^2 + y^2 \leq 1 \\
        0 & \text{elsewhere.} \end{matrix} \right.
    \]
Find $P (X \leq Y)$. 
\end{problem}

\subsection{Marginals and Independence}

\begin{problem}
Let $(X, Y)$ be a continuous random vector with joint probability differentiable density function $f_{X, Y}$. Show that
    \begin{enumerate}[label=\alph*)]
        \item $f_X (x) = \displaystyle\int_{-\infty}^\infty f_{X, Y} (x, y) \, dy$.
        \item $f_Y (y) = \displaystyle\int_{-\infty}^\infty f_{X, Y} (x, y) \, dx$.
    \end{enumerate}
\end{problem}

\begin{problem}
Let $X$ and $Y$ be two random variable with joint probability density function
    \[
        f_{X, Y} (x, y) = \left\lbrace \begin{matrix} 2 & 0 \leq y \leq x \leq 1 \\
                            0 & \text{elsewhere} \end{matrix} \right. 
    \]  
    \begin{enumerate}[label=\alph*)]
        \item Sketch $f_{X, Y}$.
        \item Are $X$, $Y$ independent?
    \end{enumerate}
\end{problem}

\begin{problem}
Let $X$ and $Y$ be two random variable with joint probability density function
    \[
        f_{X, Y} (x, y) = \left\lbrace \begin{matrix} (2y+ 1)/2 & (x, y) \in [0, 1] \times [0, 1] \\
                            0 & \text{elsewhere} \end{matrix} \right. 
    \]  
    \begin{enumerate}[label=\alph*)]
        \item Sketch $f_{X, Y}$.
        \item Are $X$, $Y$ independent?
    \end{enumerate}
\end{problem}

\begin{problem}
A bus arrives at a bus stop at a uniformly distributed time over the interval $0$ to $1$ hour. A passenger also arrives at the bus stop at a uniformly distributed time over the interval $0$ to $1$ hour. Assume that the arrival times of the bus and passenger are independent of one another and that the passenger will wait for up to $1/4$ hour for the bus to arrive. What is the probability that the passenger will catch the bus?
\end{problem}

\subsection{Important Measurements}

\begin{problem}
Prove that if $X$ and $Y$ are two independent random variables with average $\mu_X$ and $\mu_Y$, then $\mathrm{Cov} (X, Y) = 0$. 
\end{problem}

\begin{problem}
Prove that if $X$ and $Y$ are two random variables with averages $\mu_X$ and $\mu_Y$ and standard deviation $\sigma_X$ and $\sigma_Y$, then $\rho (X, Y) \in [-1, 1]$.
\end{problem}

\begin{problem}
Let $X$ and $Y$ be random variables with means $\mu_X$ and $\mu_Y$ and with variance $\sigma_X^2$ and $\sigma_Y^2$. Use the definition of the covariance to show that
    \begin{enumerate}[label=\alph*)]
        \item $\mathrm{Cov} (X, Y) = \mathrm{Cov} (Y, X)$.
        \item $\mathrm{Var} (aX + bY) = a^2 \sigma_X^2 + b^2 \sigma_Y^2 + 2 ab \mathrm{Cov} (X, Y)$. 
        \item $\mathrm{Cov} (X, X) = \sigma_X^2$.
    \end{enumerate}
\end{problem}

\begin{problem}
The random variables $X$ and $Y$ are such that $\mathrm{Exp} (X) = 4$, $\mathrm{Exp} (Y) = -1$, $\sigma_X^2 = 2$ and $\sigma_Y^2 = 8$.
    \begin{enumerate}[label=\alph*)]
            \item What is $\mathrm{Cov} (X, X)$?
            \item What is the largest possible value for $\mathrm{Cov} (X, Y)$?
    \end{enumerate}
\end{problem}

\end{document}