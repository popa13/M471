\documentclass[12pt]{article}
\usepackage[utf8]{inputenc}

\usepackage{lmodern}

\usepackage{enumitem}
\usepackage[margin=2cm]{geometry}

\usepackage{amsmath, amsfonts, amssymb}
\usepackage{graphicx}
\usepackage{subfigure}
\usepackage{tikz}
\usepackage{pgfplots}
\usepackage{multicol}

\usepackage{titlesec}
\usepackage{environ}
\usepackage{xcolor}
\usepackage{fancyhdr}
\usepackage[colorlinks = true, linkcolor = black]{hyperref}
\usepackage{xparse}
\usepackage{enumitem}
\usepackage{comment}
\usepackage{wrapfig}
\usepackage[capitalise]{cleveref}
\usepackage{epsdice}
\usepackage{circledsteps}

\usepackage{url}
\usepackage{calc}
%\usepackage{subcaption}
\usepackage[indent=0pt]{parskip}

\usepackage{array}
\usepackage{blkarray,booktabs, bigstrut}
\usepackage{bigints}

\pgfplotsset{compat=1.16}

% MATH commands
\newcommand{\ga}{\left\langle}
\newcommand{\da}{\right\rangle}
\newcommand{\oa}{\left\lbrace}
\newcommand{\fa}{\right\rbrace}
\newcommand{\oc}{\left[}
\newcommand{\fc}{\right]}
\newcommand{\op}{\left(}
\newcommand{\fp}{\right)}

\newcommand{\bi}{\mathbf{i}}
\newcommand{\bj}{\mathbf{j}}
\newcommand{\bk}{\mathbf{k}}
\newcommand{\bF}{\mathbf{F}}

\newcommand{\mR}{\mathbb{R}}
\newcommand{\mC}{\mathbb{C}}
\newcommand{\mT}{\mathbb{T}}
\newcommand{\mD}{\mathbb{D}}

\newcommand{\ra}{\rightarrow}
\newcommand{\Ra}{\Rightarrow}

\newcommand{\sech}{\mathrm{sech}\,}
\newcommand{\csch}{\mathrm{csch}\,}
\newcommand{\curl}{\mathrm{curl}\,}
\newcommand{\dive}{\mathrm{div}\,}

\newcommand{\ve}{\varepsilon}
\newcommand{\spc}{\vspace*{0.5cm}}

\DeclareMathOperator{\Ran}{Ran}
\DeclareMathOperator{\Dom}{Dom}
\DeclareMathOperator{\re}{Re}
\DeclareMathOperator{\im}{Im}
%\DeclareMathOperator{\arg}{arg}

\usepackage{pifont}

\newcommand{\club}{\ding{168}}
\newcommand{\spade}{\ding{171}}
\newcommand{\ddiamond}{{\color{red}\ding{169}}}
\newcommand{\heart}{{\color{red}\ding{170}}}

% Playing card
\newcommand{\cardH}[2]{%
\begin{tikzpicture}[trim right = #2pt, scale=0.1]
  % Card outline
  \draw[thick, red] (0,0) rectangle (5,3.5);
  % Card suit and value
  \node at (1.4,1.75) {\tiny\color{red} #1};
  \node[red] at (3.5, 1.75) {\scriptsize\heart};
\end{tikzpicture}
}
\newcommand{\cardD}[2]{%
\begin{tikzpicture}[trim right = #2pt, scale=0.1]
  % Card outline
  \draw[thick, red] (0,0) rectangle (5,3.5);
  % Card suit and value
  \node at (1,1.75) {\tiny\color{red} #1};
  \node[red] at (3.5, 1.75) {\footnotesize\ddiamond};
\end{tikzpicture}
}
\newcommand{\cardS}[2]{%
\begin{tikzpicture}[trim right = #2pt, scale=0.1]
  % Card outline
  \draw[thick, black] (0,0) rectangle (5,3.5);
  % Card suit and value
  \node at (1,1.75) {\tiny\color{black} #1};
  \node[black] at (3.5, 1.75) {\footnotesize\spade};
\end{tikzpicture}
}
\newcommand{\cardC}[2]{%
\begin{tikzpicture}[trim right = #2pt, scale=0.1]
  % Card outline
  \draw[thick, black] (0,0) rectangle (5,3.5);
  % Card suit and value
  \node at (1,1.75) {\tiny\color{black} #1};
  \node[black] at (3.5, 1.75) {\footnotesize\club};
\end{tikzpicture}
}

%% Defining example environment
\newcounter{totNumProblems}
\newcounter{problem}[section]
\NewEnviron{problem}%
	{%
	\noindent\refstepcounter{problem}\refstepcounter{totNumProblems}\fcolorbox{gray!40}{gray!40}{\textsc{\textcolor{black}{Problem~\theproblem.}}}%
	%\fcolorbox{black}{white}%
		{  %\parbox{0.95\textwidth}%
			{
			\BODY
			}%
		}%
	}


%% Redefining sections
\newcommand{\sectionformat}[1]{%
    \begin{tikzpicture}[baseline=(title.base)]
        \node[rectangle, draw] (title) {#1 \thesection};
    \end{tikzpicture}
    
    \noindent\hrulefill
}

\renewcommand{\thesection}{\Alph{section}}
\renewcommand{\thesubsection}{\Alph{section}.\arabic{subsection}}

% default values copied from titlesec documentation page 23
% parameters of \titleformat command are explained on page 4
\titleformat{\section}{\centering\normalfont\large\scshape}{}{1em}{\centering\sectionformat}

%% Set counters for sections to none
%\setcounter{secnumdepth}{0}

%% Set the footer/headers
\pagestyle{fancy}
\fancyhf{}
\renewcommand{\headrulewidth}{0pt}
\renewcommand{\footrulewidth}{2pt}
\lfoot{P.-O. Paris{\'e}}
\cfoot{MATH 471}
\rfoot{Page \thepage}

\begin{document}
\hrulefill

\begin{minipage}{0.33\textwidth}
\textsc{Math 471}
\end{minipage} \hfill 
\begin{minipage}{0.32\textwidth}
\centering
\textsc{Problems Set Solutions} \\
Appendix L
\end{minipage}
 \hfill 
 \begin{minipage}{0.33\textwidth}
 \flushright \textsc{Fall 2023}
 \end{minipage}

\hrulefill

\setcounter{section}{12}

 \subsection[~~Mathematical Statements]{Mathematical Statements}

 \begin{problem}
\begin{enumerate}[label=\alph*)]
\item Yes it is a statement. The statement is false since $|-12| = 12$ (absolute value turns negative numbers into positive numbers).
\item No, this is not a statement. The value of $x$ is not specify, so there is no truth value that can be associated to this statement.
\item No, this is not a statement. A question is not a statement.
\item Yes, this is a statement. It is true, because assuming that $a = 2$ and $b = 4$, we have $a + b = 2 + 4 = 6$.
\end{enumerate}
 \end{problem}

 \subsection[~~Logic and Mathematical Language]{Logic and Mathematical Language}

  \begin{problem}
 \begin{enumerate}[label=\alph*)]
 \item \textbf{Converse:} If Angela sleeps in, then it is a Saturday. \\
 \textbf{Contrapositive:} If Angela does not sleep in, then it is not Saturday.
 \item \textbf{Converse:} If I use my umbrella, then it rains outside. \\
 \textbf{Contrapositive:} If I don't use my umbrella, then it does not rain outside.
 \item \textbf{Converse:} If the waves are bigger than 4 foot high, then I surf.\\ 
 \textbf{Contrapositive:} If the waves are not bigger than 4 foot high, then I don't surf.
 \end{enumerate}
 \end{problem}

 \begin{problem}
 \begin{enumerate}[label=\alph*)]
 \item The negation is ``It is not the case that it is raining and Charlie is cold.''. The negation of a statement $P \wedge Q$, is $(\neg P) \vee (\neg Q)$. So, letting $P$: ``It is raining'' and $Q$: ``Charlie is cold'', a useful reformulation of the negation is ``it is not raining or Charlie is not cold''.
 \item The negation is ``It is not the case that if is raining, then Charlie is cold''. The negation of a statement $P \Rightarrow Q$ is $P \wedge (\neg Q)$. So, a useful reformulation of the negation is ``It is raining and Charlie is not cold''.
 \item Let's simplify the statement using mathematical symbols. We can equivalently and compactly rewrite the statement as `` $\forall x$ real, $\exists y$ real such that $x + y = 0$''. The negation is then ``It is not the case that $\forall x$ real, $\exists y$ real such that $x + y = 0$''. The negation of a universal statement ``$\forall x$, $P (x)$'' is ``$\exists x$, $\neg P(x)$''. Let $P (x)$: ``$\exists y$ real such that $x + y = 0$''. Then we can rewrite the negation of the statement as ``$\exists x$ real such that $\neg P (x)$'' or
 	\begin{center}
 	$\exists x$ real such that it is not the case that there exists $y$ real such that $x + y = 0$.
 	\end{center}
 The negation of an existential ``$\exists y$, $Q(y)$'' is ``$\forall y$, $\neg Q(y)$. For a fixed $x$, let $Q (y)$: ``$x + y = 0$''. Then we can rewrite the negation of ``$\exists y$ real such that $x + y = 0$'' as ``$\forall y$ real, $\neg Q (y)$'', or ``$\forall y$ real, $x + y \neq 0$''. Therefore, the negation of the whole statement is
 	\begin{center}
 	$\exists x$ real such that $\forall y$ real, $x + y \neq 0$ .
 	\end{center}
\end{enumerate}
 \end{problem}

 \begin{problem}
 	\begin{enumerate}[label=\alph*)]
 		\item Here the truth table of $P \Rightarrow Q$ and $Q \Rightarrow P$ combined. 
 		\begin{center}
 		\begin{tabular}{c|c|c|c}
 		$P$ & $Q$ & $P \Rightarrow Q$ & $Q \Rightarrow P$ \\\hline 
 		$T$ & $T$ & $T$ & $T$ \\\hline 
 		$T$ & $F$ & $F$ & $T$  \\\hline 
 		$F$ & $T$ & $T$ & $F$ \\\hline 
 		$F$ & $F$ & $T$ & $T$ \\
 		\end{tabular}
 		\end{center}
 		We see the truth value differs in the second and first rows.
 		\item Here the truth table of $P \Rightarrow Q$ and $\neg Q \Rightarrow \neg P$ combined. 
 		\begin{center}
 		\begin{tabular}{c|c|c|c|c|c}
 		$P$ & $\neg P$ & $Q$ & $\neg Q$ & $P \Rightarrow Q$ & $\neg Q \Rightarrow \neg P$ \\\hline
 		$T$ & $F$ & $T$ & $F$ & $T$ & $T$ \\\hline 
 		$T$ & $F$ & $F$ & $T$ & $F$ & $F$  \\\hline 
 		$F$ & $T$ & $T$ & $F$ & $T$ & $T$ \\\hline 
 		$F$ & $T$ & $F$ & $T$ & $T$ & $T$ \\
 		\end{tabular}
 		\end{center}
 		We see that the truth value in every rows are the same.
 	\end{enumerate}
 \end{problem}

 \subsection[~~Methods of Proof]{Methods of Proof}

 \begin{problem}
 	\begin{enumerate}[label=\alph*)]
 		\item Assume that $a$ and $b$ are odd integers. By definition, we have $a = 2k + 1$ and $b = 2l + 1$, where $k$ and $l$ are integers. Therefore
 			\begin{align*}
 			a + b = 2k + 1 + 2l + 1 = 2 (k + l + 1) .
 			\end{align*}
 		The last equation expresses $a + b$ as a multiple of $2$, so $a + b$ is even.
 		\item Assume that $a$ is even and that $b$ is odd. By the definitions, we have $a = 2k$ and $b = 2l + 1$, for some integers $k$ and $l$. Therefore
 			\begin{align*}
 			a + b = 2k + 2l + 1 = 2 (k + l) + 1 .
 			\end{align*}
 		The last equation shows that $a + b$ is an odd number.
 	\end{enumerate}
 \end{problem}

 \begin{problem}
 We will prove this by contradiction. Assume that $\sqrt{2}$ is a rational number, meaning there are two integers $p$ and $q$ such that $\sqrt{2} = p /q$. We may simplify the fraction $p/q$ so that $p$ and $q$ have no common divisors. 

 Multiplying by $q$ and squaring both sides of the equation $\sqrt{2} = p/q$ gives us the following equation
 	\begin{align*}
 	2q^2 = p^2 .
 	\end{align*}
 This means $p^2$ is even, so that $p$ is even. \textit{[If $p$ was odd, then we know that $p^2 = (p)(p)$ whould be odd, a contradiction with the fact that $p^2$ is even.]}.

 Write $p = 2k$, for some integer $k$. Replacing the new expression of $p$ is the last equation and after simplifying, we obtain
 	\begin{align*}
 	q^2 = 2k^2 .
 	\end{align*}
 Therefore, $q^2$ is even, so that $q$ is even. But if $p$ and $q$ are even, they share a common divisor, that is $2$. But we assumed that $p$ and $q$ have no common divisors and this is a contradiction. 

 Therefore, $\sqrt{2}$ is not a rational number.
 \end{problem}

 \begin{problem}
 Set $m = 3$ and $n = 2$, so that $(2)(3) + (3)(2) = 6 + 6 = 12$.
 \end{problem}

\end{document}