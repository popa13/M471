\documentclass[12pt]{article}
\usepackage[utf8]{inputenc}

\usepackage{lmodern}

\usepackage{enumitem}
\usepackage[margin=2cm]{geometry}

\usepackage{amsmath, amsfonts, amssymb}
\usepackage{graphicx}
\usepackage{subfigure}
\usepackage{tikz}
\usepackage{pgfplots}
\usepackage{multicol}

\usepackage{titlesec}
\usepackage{environ}
\usepackage{xcolor}
\usepackage{fancyhdr}
\usepackage[colorlinks = true, linkcolor = black]{hyperref}
\usepackage{xparse}
\usepackage{enumitem}
\usepackage{comment}
\usepackage{wrapfig}
\usepackage[capitalise]{cleveref}
\usepackage{epsdice}
\usepackage{circledsteps}

\usepackage{url}
\usepackage{calc}
%\usepackage{subcaption}
\usepackage[indent=0pt]{parskip}

\usepackage{array}
\usepackage{blkarray,booktabs, bigstrut}
\usepackage{bigints}

\pgfplotsset{compat=1.16}

% MATH commands
\newcommand{\ga}{\left\langle}
\newcommand{\da}{\right\rangle}
\newcommand{\oa}{\left\lbrace}
\newcommand{\fa}{\right\rbrace}
\newcommand{\oc}{\left[}
\newcommand{\fc}{\right]}
\newcommand{\op}{\left(}
\newcommand{\fp}{\right)}

\newcommand{\bi}{\mathbf{i}}
\newcommand{\bj}{\mathbf{j}}
\newcommand{\bk}{\mathbf{k}}
\newcommand{\bF}{\mathbf{F}}

\newcommand{\mR}{\mathbb{R}}
\newcommand{\mC}{\mathbb{C}}
\newcommand{\mT}{\mathbb{T}}
\newcommand{\mD}{\mathbb{D}}

\newcommand{\ra}{\rightarrow}
\newcommand{\Ra}{\Rightarrow}

\newcommand{\sech}{\mathrm{sech}\,}
\newcommand{\csch}{\mathrm{csch}\,}
\newcommand{\curl}{\mathrm{curl}\,}
\newcommand{\dive}{\mathrm{div}\,}

\newcommand{\ve}{\varepsilon}
\newcommand{\spc}{\vspace*{0.5cm}}

\DeclareMathOperator{\Ran}{Ran}
\DeclareMathOperator{\Dom}{Dom}
\DeclareMathOperator{\re}{Re}
\DeclareMathOperator{\im}{Im}
%\DeclareMathOperator{\arg}{arg}

\usepackage{pifont}

\newcommand{\club}{\ding{168}}
\newcommand{\spade}{\ding{171}}
\newcommand{\ddiamond}{{\color{red}\ding{169}}}
\newcommand{\heart}{{\color{red}\ding{170}}}

% Playing card
\newcommand{\cardH}[2]{%
\begin{tikzpicture}[trim right = #2pt, scale=0.1]
  % Card outline
  \draw[thick, red] (0,0) rectangle (5,3.5);
  % Card suit and value
  \node at (1.4,1.75) {\tiny\color{red} #1};
  \node[red] at (3.5, 1.75) {\scriptsize\heart};
\end{tikzpicture}
}
\newcommand{\cardD}[2]{%
\begin{tikzpicture}[trim right = #2pt, scale=0.1]
  % Card outline
  \draw[thick, red] (0,0) rectangle (5,3.5);
  % Card suit and value
  \node at (1,1.75) {\tiny\color{red} #1};
  \node[red] at (3.5, 1.75) {\footnotesize\ddiamond};
\end{tikzpicture}
}
\newcommand{\cardS}[2]{%
\begin{tikzpicture}[trim right = #2pt, scale=0.1]
  % Card outline
  \draw[thick, black] (0,0) rectangle (5,3.5);
  % Card suit and value
  \node at (1,1.75) {\tiny\color{black} #1};
  \node[black] at (3.5, 1.75) {\footnotesize\spade};
\end{tikzpicture}
}
\newcommand{\cardC}[2]{%
\begin{tikzpicture}[trim right = #2pt, scale=0.1]
  % Card outline
  \draw[thick, black] (0,0) rectangle (5,3.5);
  % Card suit and value
  \node at (1,1.75) {\tiny\color{black} #1};
  \node[black] at (3.5, 1.75) {\footnotesize\club};
\end{tikzpicture}
}

%% Defining example environment
\newcounter{totNumProblems}
\newcounter{problem}[section]
\NewEnviron{problem}%
	{%
	\noindent\refstepcounter{problem}\refstepcounter{totNumProblems}\fcolorbox{gray!40}{gray!40}{\textsc{\textcolor{black}{Problem~\theproblem.}}}%
	%\fcolorbox{black}{white}%
		{  %\parbox{0.95\textwidth}%
			{
			\BODY
			}%
		}%
	}


%% Redefining sections
\newcommand{\sectionformat}[1]{%
    \begin{tikzpicture}[baseline=(title.base)]
        \node[rectangle, draw] (title) {#1 \thesection};
    \end{tikzpicture}
    
    \noindent\hrulefill
}

\renewcommand{\thesection}{\Alph{section}}
\renewcommand{\thesubsection}{\Alph{section}.\arabic{subsection}}

% default values copied from titlesec documentation page 23
% parameters of \titleformat command are explained on page 4
\titleformat{\section}{\centering\normalfont\large\scshape}{}{1em}{\centering\sectionformat}

%% Set counters for sections to none
%\setcounter{secnumdepth}{0}

%% Set the footer/headers
\pagestyle{fancy}
\fancyhf{}
\renewcommand{\headrulewidth}{0pt}
\renewcommand{\footrulewidth}{2pt}
\lfoot{P.-O. Paris{\'e}}
\cfoot{MATH 471}
\rfoot{Page \thepage}

\begin{document}
\hrulefill

\begin{minipage}{0.33\textwidth}
\textsc{Math 471}
\end{minipage} \hfill 
\begin{minipage}{0.32\textwidth}
\centering
\textsc{Solutions to Problems} \\
Chapter G
\end{minipage}
 \hfill 
 \begin{minipage}{0.33\textwidth}
 \flushright \textsc{Fall 2023}
 \end{minipage}

\hrulefill

\setcounter{section}{7}

\subsection{Mean-Square Law of Large Numbers}

\begin{problem}
We have
    \[
        \mathrm{Exp} ((aX_n + b - (aX + b))^2) = \mathrm{Exp} (a^2 (X_n - X)^2) = a^2 \mathrm{Exp} ( (X_n - X)^2 ) .
    \]
By assumption, $\lim_{n \ra \infty} \mathrm{Exp} ( (X_n - X)^2) = 0$, therefore
    \[
        \lim_{n \ra \infty} \mathrm{Exp} ( (aX_n + b - (aX + b))^2) = a^2 \lim_{n \ra \infty} \mathrm{Exp} ( (X_n - X)^2 ) = 0 . 
    \]
Hence, $aX_n + b \ra aX + b$ in mean-square. \hfill $\triangle$
\end{problem}

\begin{problem}
Let $N_m$ be the number of occurences of $5$ or $6$ in $m$ throws of a fair die. Show that
    \[
        \frac{1}{m} N_m \ra \frac{1}{3} \quad \text{ in mean square}
     \] 
as $m \ra \infty$.

Let $X_i$ be the random variable with output $1$ if the die lands on $5$ or $6$ and output $0$ if the die lands on $1, 2, 3$, or $4$. Then, we have
    \[
        \mathrm{Exp} (X_i) = 0 \times \frac{2}{3} + 1 \times \frac{1}{3} = \frac{1}{3}
    \]
and
    \[
        \mathrm{Var} (X_i ) = \mathrm{Exp} (X_i^2) - (\mathrm{Exp} (X_i))^2 = \frac{1}{3} - \frac{1}{9} = \frac{2}{9} .
    \]
Therefore, we get $\mathrm{Exp} (N_m) = \frac{m}{3}$ and $\mathrm{Var} (N_m ) = \frac{2m}{9}$ because in $N_m$ are independent. Hence, we compute
    \[
        \mathrm{Exp} \Big( \Big( \frac{N_m}{m} - \frac{1}{3} \Big)^2\Big) = \mathrm{Exp} \Big( \frac{(N_m - \frac{m}{3})^2}{m^2} \Big) = \frac{1}{m^2} \mathrm{Var} (N_m) = \frac{2}{9m} .
    \]
As $m \ra \infty$, $\frac{2}{9m} \ra 0$ and therefore $N_m/m \ra 1/3$ in mean-square, as $m \ra \infty$. \hfill $\triangle$
\end{problem}

%\begin{problem}
%Prove the following alternative form of Chebyshev's inequality: If $X$ is a random variable with finite variance and $a > 0$, then
%   \[
%       P \big( |X - \mathrm{Exp} (X)| > a \big) \leq \frac{1}{a^2} \mathrm{Var} (X) .
%   \]
%\end{problem}

%\begin{problem}
%Show that if $X_n \ra X$ in probability, then, as $n \ra \infty$,
%   \[
%       aX_n + b \ra aX + b \quad \text{in probability},
%   \]
%for any $a, b \in \mR$.
%\end{problem}

\subsection{Central Limit Theorem}

\begin{problem}

    \begin{enumerate}[label=\alph*)]
    \item Let $X_i$ be a random variable with output the facture strenght of the $i$-th piece. Then, we have $\mu = \mathrm{Exp} (X_i) = 14$, for any $i$ and $\sigma^2 = \mathrm{Var} (X_i) = 4$ for any $i$. We have $n = 100$, the size of the sample and let $S_n/n = (X_1 + X_2 + \ldots + X_n)/n$ represents the average of the fracture strength in the sample.

    We will use the Central Limit Theorem to estimate the probability
        \[
            P \Big( \frac{S_{100}}{100} > 14.5 \Big) .
        \]
    The standardized version of $S_{100}$ is 
        \[
            Z_{100} = \frac{S_{100} - 100 \mu}{\sqrt{100} \sigma} = \frac{S_{100} - 1400}{20} .
        \]
    We see that
        \[
            \frac{S_{100}}{100} > 14.5 \iff S_{100} > 1450 \iff S_{100} - 1400 > 50 \iff \frac{S_{100} - 1400}{20} > 2.5 .
        \]
    Therefore, 
        \[
            P \Big( \frac{S_{100}}{100} > 14.5 \Big) = P (Z_{100} > 2.5 ) .
        \]
    
    From the Central Limit Theorem,
        \[
            P \Big( \frac{S_{100}}{100} > 14.5 \Big) = P (Z_{100} > 2.5 ) \approx P (Z > 2.5 )
        \]
    where $Z \sim N (0, 1 )$. Using the table of the normal distribution, we find that
        \[
            P (Z > 2.5) = 1 - P (Z \leq 2.5 ) = 1 - 0.99379 = 0.00731 .
        \]
    \item Since the standardized version is centered at the average, we will try to find $a$ such that $S_n/n \in [\mu - a , \mu + a ]$ in $95\%$ of the chances. We want to find $a > 0$ such that
        \[
            P \Big( \left| \frac{S_{100}}{100} - \mu \right| < a \Big) = 0.95 .
        \]
    Rearranging the left-hand side:
        \[
            \frac{S_{100}}{100} - 14 = \frac{S_{100} - 1400}{100} = \frac{2}{10} \Big( \frac{S_{100} - 1400}{20} \Big) = \frac{2}{10} Z_n
        \]
    and therefore
        \[
            P \Big( \left| \frac{S_{100}}{100} - 1400 \right| < a \Big) = P (|Z_n| < 5a ) .
        \]
    Using the Central Limit Theorem, $P (|Z_n| < 5a ) \approx P (|Z| < 5a )$, for $Z \sim N (0, 1)$ and we have to find $a$ such that $P (|Z| < 5a ) = 0.95$. Now, since the normal density of $N (0, 1)$ is symmetric with respect to the $y$-axis, we have $P (Z > 5a ) = 0.025 = P (Z < -5a )$. Therefore, 
        \[
            P (|Z| < 5a ) = 0.95 \iff P (Z < 5a ) = 0.975 .
        \]
    Using the table, we find $z = 1.96$ and therefore $a = 1.96/5 = 0.392$. Hence, the interval containing $S_{100}/100$ in $95\%$ of the chances is $[13.608, 14.392]$. 
    \end{enumerate}
\end{problem}

\end{document}